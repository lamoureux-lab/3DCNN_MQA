\documentclass[letter,10pt]{article}
\usepackage[utf8]{inputenc}
\usepackage[english]{babel}
\usepackage{float}
\usepackage{graphicx}
\usepackage{caption}
\usepackage{subcaption}
\usepackage{amsmath}
\usepackage{multirow}
\usepackage{color}
\usepackage{tabularx}
\usepackage{geometry}
\usepackage{capt-of}
% \renewcommand\floatpagefraction{0.9}
\title{Supporting information for:\\ Deep convolutional networks for quality assessment of protein folds}
\author{}

\renewcommand*{\thefigure}{S\arabic{figure}}
\renewcommand*{\thetable}{S\arabic{table}}

\begin{document}

\maketitle


\begin{figure}[H]
    \centering
    \includegraphics[width=\linewidth]{Fig/datasetLengthDistributions.png}
    \caption{Distributions of sequence lengths for targets in training set (blue) and test set (red).}
    \label{Fig:dataLengthDist}
\end{figure}

\begin{table}[H]
\begin{center}
%
    \caption{Closest homolog sequences from the training set.
      sequence pair is reported if at least one training sequence
      aligns to a test sequence with a blastp E-value less than
      $10^{-4}$. Only the top alignment is reported for each test
      sequence.}
%
\begin{tabular}{ c | c | l }
    
    Test set ID & Closest training set ID & E-value \\
    \hline
    T0768 & T0690 & $2.70\times 10^{-13}$ \\
    T0770 & T0645 & $1.79\times 10^{-13}$ \\
    T0772 & T0518 & $1.89\times 10^{-7}$ \\
    T0776 & T0707 & $3.98\times 10^{-5}$ \\
    T0783 & T0699 & $1.19\times 10^{-22}$ \\
    T0798 & T0308 & $9.57\times 10^{-6}$ \\
    T0813 & T0398 & $2.45\times 10^{-5}$ \\
    T0819 & T0636 & $8.66\times 10^{-15}$ \\
    T0854 & T0324 & $2.13\times 10^{-13}$ \\
\end{tabular}
\label{Tbl:datasetsSimilarity}
\end{center}
\end{table}


\begin{table}[H]
\begin{center}
\caption{Targets from test and training sets that belong to the same
  Pfam family \cite{finn2016pfam}, based on a HMMER search
  \cite{finn2015hmmer} with an E-value cutoff of 1.0. With that
  cutoff, 403 of the 564 training targets and 65 of the 83 test
  targets could be assigned families. There are 25 families containing
  at least one test sequence and one training sequence, involving a
  total of 16 test targets and 42 training targets. Each of the 25
  families belongs to a distinct Pfam clan.}
%
\begin{tabular}{ l | l | l }

    Common & Test set target & Training set targets \\
    family & & \\
    \hline
    PF00795 & T0794 & T0542 \\ \hline
    PF13472 & T0776 & T0448, T0297, T0286, T0750 \\ \hline
    PF03807 & T0813 & T0398, T0393, T0702 \\ \hline
    PF00266 & T0801 & T0339, T0697 \\ \hline
    PF01128 & T0783 & T0699, T0420 \\ \hline
    PF07949 & T0780 & T0572 \\ \hline
    PF13577 & T0815 & T0752, T0736 \\ \hline
    PF12804 & T0783 & T0593, T0699, T0420 \\ \hline
    PF13242 & T0854 & T0371, T0341, T0303, T0324, T0330, T0329, T0418 \\ \hline
    PF13306 & T0768 & T0690, T0671, T0713, T0653 \\ \hline
    PF12741 & T0770 & T0664, T0645, T0532 \\ \hline
    PF00025 & T0798 & T0308 \\ \hline
    PF12872 & T0792 & T0549 \\ \hline
    PF03446 & T0813, T0851 & T0398, T0393, T0702 \\ \hline
    PF00155 & T0801, T0819 & T0591, T0636, T0436, T0697 \\ \hline
    PF13419 & T0854 & T0371, T0341, T0303, T0379, T0324, T0330, T0329, T0418, T0635 \\ \hline
    PF12680 & T0815 & T0451, T0475 \\ \hline
    PF06439 & T0772 & T0518 \\ \hline
    PF12771 & T0770 & T0664, T0645, T0532 \\ \hline
    PF08477 & T0798 & T0308 \\ \hline
    PF00657 & T0776 & T0297, T0286, T0679 \\ \hline
    PF00071 & T0798 & T0308 \\ \hline
    PF00702 & T0854 & T0303, T0324, T0330, T0329, T0418, T0635 \\ \hline
    PF01926 & T0798 & T0308 \\ \hline
    PF12697 & T0764 & T0672 \\ \hline
\end{tabular}
\label{Tbl:SharedPfam}
\end{center}
\end{table}

% \begin{table}[H]
% \begin{center}
%
% \caption{Targets from CASP11 that do not have corresponding PDB
%   entries or ECOD classes.}
%
% \begin{tabular}{ l | l | l }

%     Target & Has PDB & Has ECOD class \\
%     \hline
%     T0773 &True &False\\
%     T0820 &False &False\\
%     T0823 &False &False\\
%     T0824 &False &False\\
%     T0827 &False &False\\
%     T0835 &False &False\\
%     T0836 &False &False\\
%     T0838 &False &False\\
% \end{tabular}
% \label{Tbl:CASP11PDB_ECOD}
% \end{center}
% \end{table}


\begin{figure}[H]
    \centering
    \includegraphics[width=\linewidth]{Fig/folds_graph.png}
%
    \caption{Classification of the test set structures into the top
      four ECOD structural levels (from the center out): architecture
      (A), possible homology (X), homology (H), and topology (T). The
      names of the architecture types are shown in the outer circle of
      the diagram. Branches drawn in black correspond to groups that
      have representatives in both training and test sets. Branches
      drawn in grey correspond to groups unique to the test set.
%
      Four architecture groups present overlap at all levels between
      the training and test sets: ``a/b barrels'', ``beta duplicates
      or obligate multimers'', ``a+b complex topology'', and ``a+b
      four layers''.
%
      We do not show the F-groups because they have litle overlap
      among the training and test sets.
%
      Targets T0773, T0797, and T0816 are excluded from the analysis
      because they have no ECOD classification, and targets T0820,
      T0823, T0824, T0827, T0835, and T0836 are excluded because they
      have no structure in the RCSB PDB.}
%
    \label{Fig:foldsGraph}
\end{figure}

\begin{table}[H]
\begin{center}
%
\caption{Details of the model architecture. Layer 10 is marked with an
  asterisk to indicate that its output was used in the Grad-CAM
  analysis reported in Section~4 of the main text.}
%
\makebox[0pt][c]{
\hskip-\footskip
\begin{tabularx}{0.8\paperwidth}{ l | l | c | c | c }

    Layer & Type & Input dimensions & Output dimensions & Parameters \\
    \hline
    1&3D Convolution & $11\times 120\times 120\times 120$ & $16\times 118\times 118\times 118$ & 
                    Filter size $3\times 3\times 3$, stride $1$\\
    2&ReLU & & & \\
    3&Max pooling & $16\times 118\times 118\times 118$ & $16\times 58\times 58\times 58$ & 
                    Filter size $3\times 3\times 3$, stride $2$ \\
    \hline 
    4&3D Convolution & $16\times 58\times 58\times 58$ & $32\times 56\times 56\times 56$ & 
                    Filter size $3\times 3\times 3$, stride $1$\\
    5&Batch normalization & & & \\
    6&ReLU & & & \\
    7&Max pooling & $32\times 56\times 56\times 56$ & $32\times 27\times 27\times 27$ & 
                    Filter size $3\times 3\times 3$, stride $2$ \\
    \hline
    8&3D Convolution & $32\times 27\times 27\times 27$ & $32\times 25\times 25\times 25$ & 
                    Filter size $3\times 3\times 3$, stride $1$\\
    9&Batch normalization & & & \\
    10*&ReLU & & & \\
    11&3D Convolution & $32\times 25\times 25\times 25$ & $64\times 23\times 23\times 23$ & 
                    Filter size $3\times 3\times 3$, stride $1$\\
    12&Batch normalization & & & \\
    13&ReLU & & & \\
    14&Max pooling & $64\times 23\times 23\times 23$ & $64\times 11\times 11\times 11$ & 
                    Filter size $3\times 3\times 3$, stride $2$ \\
    \hline
    15&3D Convolution & $64\times 11\times 11\times 11$ & $128\times 9\times 9\times 9$ & 
                    Filter size $3\times 3\times 3$, stride $1$\\
    16&Batch normalization & & & \\
    17&ReLU & & & \\
    18&3D Convolution & $128\times 9\times 9\times 9$ & $128\times 7\times 7\times 7$ & 
                    Filter size $3\times 3\times 3$, stride $1$\\
    19&Batch normalization & & & \\
    20&ReLU & & & \\
    21&3D Convolution & $128\times 7\times 7\times 7$ & $256\times 5\times 5\times 5$ & 
                    Filter size $3\times 3\times 3$, stride $1$\\
    22&Batch normalization & & & \\
    23&ReLU & & & \\
    24&3D Convolution & $256\times 5\times 5\times 5$ & $512\times 3\times 3\times 3$ & 
                    Filter size $3\times 3\times 3$, stride $1$\\
    25&Batch normalization & & & \\
    26&ReLU & & & \\
    27&Max pooling & $512\times 3\times 3\times 3$ & $512\times 1\times 1\times 1$ & 
                    Filter size $3\times 3\times 3$, stride $2$ \\
    \hline
    28&Reshape & $512\times 1\times 1\times 1$ & $512$ & \\
    \hline
    29&Linear & 512 & 256 & \\
    30&ReLU & & & \\
    31&Linear & 256 & 128 & \\
    32&ReLU & & & \\
    33&Linear & 128 & 1 & \\
    \hline

\end{tabularx}
\hskip\headheight
}
\label{Tbl:SuppModel}
\end{center}
\end{table}


\pagebreak
\section*{Training procedure}

\subsection*{Rotations}
The rotations were sampled in the following way:
\begin{enumerate}
\item Select three random uniform values $u_1, u_2, u_3 \in [0,1]$
\item Construct quaternion from them
\begin{eqnarray}
q_0 = \sqrt(1-u_1) \sin(2\pi  u_2) \\
q_1 = \sqrt(1-u_1) \cos(2\pi  u_2) \\
q_2 = \sqrt(u_1) \sin(2\pi  u_3) \\
q_3 = \sqrt(u_1) \cos(2\pi  u_3) 
\end{eqnarray}
\item Construct the corresponding rotation matrix
\end{enumerate}

\subsection*{Translations}
The translations were sampled according to the following algorithm:
\begin{enumerate}
\item Compute the bounding box of the atoms in the decoy
\item Translate the protein so that the center of the bounding box corresponds to the frame origin 
\item Rotate protein using random uniform rotation
\item Calculate the maximum shift along each axis:
$$
D_k = \tfrac{1}{2}\max\left[ 0, \tfrac{1}{2}S_k - \tfrac{1}{2}B_k \right], \quad k \in {x, y, z}
$$
where $S_k$ are input size along the axes x,y,z and $B_k$ are the bounding box sizes along the same axes.

\item Sample random uniform translation along each axis from the
  interval $[-\tfrac{1}{2}D_k, \tfrac{1}{2}D_k], k \in {x,y,z}$
\end{enumerate}

\subsection*{Training parameters}

The loss function is optimized using the Adam algorithm \cite{kingma2014adam},
with a learning rate of $0.0003$ and a learning rate decay of $0.01$.

% \pagebreak
\begin{figure}[H]

    \centering
    \makebox[0pt][c]{
    \hskip-\footskip
    \includegraphics[width=0.8\paperwidth]{Fig/epoch40_funnels.png}
    \hskip\headheight
    }

    \caption{Scoring funnels on the validation set at epoch 40. The
      $x$-axis is the GDT\_TS score and the $y$-axis is the $f$
      score. The score was sampled once using random rotation and
      translation.}
%
    \label{Fig:ValidationEpoch40}
\end{figure}


%\section{Results}

\begin{figure}[H]
    \centering
    \includegraphics[width=\linewidth]{Fig/sampling_dist.eps}
%
    \caption{Distribution of the score of decoy FALCON\_EnvFold\_TS1
    for target T0832 under random translations and rotations. The
    distribution fits a normal distribution with an average $\mu =
    2.74$ and a standard deviation $\sigma = 0.37$ (shown in green
    dashed lines). The figure also shows the distributions of the
    score under rotations only (red lines) and under translations only
    (black dashed lines).}
%
    \label{Fig:ScoreDistribution}
\end{figure}


\begin{figure}[H]
    \centering
    \includegraphics[width=\linewidth]{Fig/LossVsECOD.png}
%
    \caption{Per-target average loss of the MQA algorithms of Table~3
    on the CASP11 test set stage~2, divided into 5 subsets of
    increasing structural similarity with the training set. The
    subsets are chosen according to the presence in the training set
    of structures belonging to the same ECOD categories (see text for
    details). 
    ``No overlap'', the structures for which there is no structure in the training set with 
    the same ECOD (T0797 and T0773). 
    ``A'', the structures in the same A-group of at least one training
    structure but not in the same X-group (T0759, T0763, T0769, etc.); 
    ``A+X'', the structures in the same X-group of at least one training
    structure but not in the same H-group (T0760, T0761, T0765, etc.);
    ``A+X+H+T'', the structures in the same T-group of at least one
    training structure but not in the same F-group (T0762, T0766, T0767,
    etc.);
    ``A+X+H+T+F'', the structures in the same F-group of at least one
    training structure (T0764, T0768, T0770, etc.);
    Error bars show per-target standard error of mean of the loss.}
%
    \label{Fig:LossVsECOD}
\end{figure}

\begin{figure}[H]
    \centering
    \makebox[0pt][c]{
    \hskip-\footskip
    \includegraphics[width=0.9\paperwidth]{Fig/CASP11Stage1_SCWRL_sFinal_funnels.png}
    \hskip\headheight
    }
    \caption{Scoring funnels on Stage 1 CASP11}
    \label{Fig:Satage1CASP11Funnels}
\end{figure}

\begin{figure}[H]
    \centering
    \makebox[0pt][c]{
    \hskip-\footskip
    \includegraphics[width=0.9\paperwidth]{Fig/CASP11Stage2_SCWRL_sFinal_funnels.png}
    \hskip\headheight
    }
    \caption{Scoring funnels on Stage 2 CASP11}
    \label{Fig:Satage2CASP11Funnels}
\end{figure}

\begin{figure}[H]
    \centerline{\includegraphics[width=0.8\linewidth]{Fig/FigT0776.eps}}
    \caption{Left: Output of the Grad-CAM analysis for layer 10 of
      the network projected on candidate structure Distill\_TS3 of
      target T0776. The isosurface shows the scaled outputs at the
      two-sigma level. The intensities of the outputs at the positions
      of the protein atoms are color-coded on the cartoon rendering of
      the structure, from blue (low intensity) to red (high
      intensity). Right: Cartoon representation of the Distill\_TS3
      decoy structure (in blue) aligned to the native structure (in
      red).}
    \label{Fig:GradCAMT0776}
\end{figure}



%\documentclass[letter,10pt]{article}
%\usepackage{graphicx}
%\usepackage{tabularx}
%\usepackage{ltxtable}
%\begin{document}
\captionof{table}{Output of the Grad-CAM algorithm on the selected representative decoys in CASP11 Stage2.
The interval of all the decoys GDT\_TS was divided into four bins of equal size and random decoys from each bin were 
selected. Then 30 samples with random rotations and translations for each selected decoys were used to generate Grad-CAM
output dencity maps. Afterwards, each map was projected on the atoms of each sampled decoy. Finally, average values of the 
projected values were calculated for each atom. These values were plotted using Pymol with the rainbow color scheme.}
		
	\begin{center}
	\makebox[10pt][c]{
	\hskip-\footskip
	\begin{tabularx}{0.95\paperwidth}{X*{4}{p{5.0cm}}}
					\hline
\tiny{T0759} &\tiny{T0759} &\tiny{T0759} &\tiny{T0759} \\
\tiny{myprotein-me\_TS4} &\tiny{BAKER-ROSETTASERVER\_TS4} &\tiny{FFAS03\_TS1} &\tiny{Seok-server\_TS2} \\
\tiny{GDT\_TS = 0.25} &\tiny{GDT\_TS = 0.31} &\tiny{GDT\_TS = 0.34} &\tiny{GDT\_TS = 0.36} \\
\begin{minipage}{\linewidth}\includegraphics[width=\linewidth]{GradCAMOutput/T0759/T0759_myprotein-me_TS4.png}\end{minipage}&\begin{minipage}{\linewidth}\includegraphics[width=\linewidth]{GradCAMOutput/T0759/T0759_BAKER-ROSETTASERVER_TS4.png}\end{minipage}&\begin{minipage}{\linewidth}\includegraphics[width=\linewidth]{GradCAMOutput/T0759/T0759_FFAS03_TS1.png}\end{minipage}&\begin{minipage}{\linewidth}\includegraphics[width=\linewidth]{GradCAMOutput/T0759/T0759_Seok-server_TS2.png}\end{minipage}\\
\hline
\tiny{T0760} &\tiny{T0760} &\tiny{T0760} &\tiny{T0760} \\
\tiny{FALCON\_MANUAL\_TS2} &\tiny{Seok-server\_TS2} &\tiny{MULTICOM-CONSTRUCT\_TS1} &\tiny{FFAS03\_TS1} \\
\tiny{GDT\_TS = 0.49} &\tiny{GDT\_TS = 0.55} &\tiny{GDT\_TS = 0.60} &\tiny{GDT\_TS = 0.62} \\
\begin{minipage}{\linewidth}\includegraphics[width=\linewidth]{GradCAMOutput/T0760/T0760_FALCON_MANUAL_TS2.png}\end{minipage}&\begin{minipage}{\linewidth}\includegraphics[width=\linewidth]{GradCAMOutput/T0760/T0760_Seok-server_TS2.png}\end{minipage}&\begin{minipage}{\linewidth}\includegraphics[width=\linewidth]{GradCAMOutput/T0760/T0760_MULTICOM-CONSTRUCT_TS1.png}\end{minipage}&\begin{minipage}{\linewidth}\includegraphics[width=\linewidth]{GradCAMOutput/T0760/T0760_FFAS03_TS1.png}\end{minipage}\\
\hline
\tiny{T0761} &\tiny{T0761} &\tiny{T0761} &\tiny{T0761} \\
\tiny{MULTICOM-CONSTRUCT\_TS1} &\tiny{RBO\_Aleph\_TS3} &\tiny{RBO\_Aleph\_TS4} &\tiny{BAKER-ROSETTASERVER\_TS4} \\
\tiny{GDT\_TS = 0.09} &\tiny{GDT\_TS = 0.10} &\tiny{GDT\_TS = 0.12} &\tiny{GDT\_TS = 0.16} \\
\begin{minipage}{\linewidth}\includegraphics[width=\linewidth]{GradCAMOutput/T0761/T0761_MULTICOM-CONSTRUCT_TS1.png}\end{minipage}&\begin{minipage}{\linewidth}\includegraphics[width=\linewidth]{GradCAMOutput/T0761/T0761_RBO_Aleph_TS3.png}\end{minipage}&\begin{minipage}{\linewidth}\includegraphics[width=\linewidth]{GradCAMOutput/T0761/T0761_RBO_Aleph_TS4.png}\end{minipage}&\begin{minipage}{\linewidth}\includegraphics[width=\linewidth]{GradCAMOutput/T0761/T0761_BAKER-ROSETTASERVER_TS4.png}\end{minipage}\\
\hline
\tiny{T0762} &\tiny{T0762} &\tiny{T0762} &\tiny{T0762} \\
\tiny{BhageerathH\_TS4} &\tiny{RBO\_Aleph\_TS3} &\tiny{MULTICOM-CONSTRUCT\_TS1} &\tiny{FFAS03\_TS1} \\
\tiny{GDT\_TS = 0.70} &\tiny{GDT\_TS = 0.76} &\tiny{GDT\_TS = 0.79} &\tiny{GDT\_TS = 0.83} \\
\begin{minipage}{\linewidth}\includegraphics[width=\linewidth]{GradCAMOutput/T0762/T0762_BhageerathH_TS4.png}\end{minipage}&\begin{minipage}{\linewidth}\includegraphics[width=\linewidth]{GradCAMOutput/T0762/T0762_RBO_Aleph_TS3.png}\end{minipage}&\begin{minipage}{\linewidth}\includegraphics[width=\linewidth]{GradCAMOutput/T0762/T0762_MULTICOM-CONSTRUCT_TS1.png}\end{minipage}&\begin{minipage}{\linewidth}\includegraphics[width=\linewidth]{GradCAMOutput/T0762/T0762_FFAS03_TS1.png}\end{minipage}\\

\end{tabularx}
%}
\hskip\headheight}
\end{center}	
	\begin{center}
	\makebox[10pt][c]{
	\hskip-\footskip
	\begin{tabularx}{0.95\paperwidth}{X*{4}{p{5.0cm}}}
					\hline
\tiny{T0763} &\tiny{T0763} &\tiny{T0763} &\tiny{T0763} \\
\tiny{FALCON\_MANUAL\_TS2} &\tiny{Distill\_TS3} &\tiny{RBO\_Aleph\_TS3} &\tiny{MULTICOM-CONSTRUCT\_TS1} \\
\tiny{GDT\_TS = 0.12} &\tiny{GDT\_TS = 0.13} &\tiny{GDT\_TS = 0.15} &\tiny{GDT\_TS = 0.17} \\
\begin{minipage}{\linewidth}\includegraphics[width=\linewidth]{GradCAMOutput/T0763/T0763_FALCON_MANUAL_TS2.png}\end{minipage}&\begin{minipage}{\linewidth}\includegraphics[width=\linewidth]{GradCAMOutput/T0763/T0763_Distill_TS3.png}\end{minipage}&\begin{minipage}{\linewidth}\includegraphics[width=\linewidth]{GradCAMOutput/T0763/T0763_RBO_Aleph_TS3.png}\end{minipage}&\begin{minipage}{\linewidth}\includegraphics[width=\linewidth]{GradCAMOutput/T0763/T0763_MULTICOM-CONSTRUCT_TS1.png}\end{minipage}\\
\hline
\tiny{T0764} &\tiny{T0764} &\tiny{T0764} &\tiny{T0764} \\
\tiny{RBO\_Aleph\_TS4} &\tiny{BioShell-server\_TS3} &\tiny{FFAS03\_TS1} &\tiny{MUFOLD-Server\_TS4} \\
\tiny{GDT\_TS = 0.50} &\tiny{GDT\_TS = 0.60} &\tiny{GDT\_TS = 0.65} &\tiny{GDT\_TS = 0.72} \\
\begin{minipage}{\linewidth}\includegraphics[width=\linewidth]{GradCAMOutput/T0764/T0764_RBO_Aleph_TS4.png}\end{minipage}&\begin{minipage}{\linewidth}\includegraphics[width=\linewidth]{GradCAMOutput/T0764/T0764_BioShell-server_TS3.png}\end{minipage}&\begin{minipage}{\linewidth}\includegraphics[width=\linewidth]{GradCAMOutput/T0764/T0764_FFAS03_TS1.png}\end{minipage}&\begin{minipage}{\linewidth}\includegraphics[width=\linewidth]{GradCAMOutput/T0764/T0764_MUFOLD-Server_TS4.png}\end{minipage}\\
\hline
\tiny{T0765} &\tiny{T0765} &\tiny{T0765} &\tiny{T0765} \\
\tiny{MUFOLD-Server\_TS4} &\tiny{RBO\_Aleph\_TS3} &\tiny{Distill\_TS3} &\tiny{MULTICOM-CONSTRUCT\_TS1} \\
\tiny{GDT\_TS = 0.19} &\tiny{GDT\_TS = 0.31} &\tiny{GDT\_TS = 0.39} &\tiny{GDT\_TS = 0.42} \\
\begin{minipage}{\linewidth}\includegraphics[width=\linewidth]{GradCAMOutput/T0765/T0765_MUFOLD-Server_TS4.png}\end{minipage}&\begin{minipage}{\linewidth}\includegraphics[width=\linewidth]{GradCAMOutput/T0765/T0765_RBO_Aleph_TS3.png}\end{minipage}&\begin{minipage}{\linewidth}\includegraphics[width=\linewidth]{GradCAMOutput/T0765/T0765_Distill_TS3.png}\end{minipage}&\begin{minipage}{\linewidth}\includegraphics[width=\linewidth]{GradCAMOutput/T0765/T0765_MULTICOM-CONSTRUCT_TS1.png}\end{minipage}\\
\hline
\tiny{T0766} &\tiny{T0766} &\tiny{T0766} &\tiny{T0766} \\
\tiny{RBO\_Aleph\_TS3} &\tiny{TASSER-VMT\_TS2} &\tiny{MULTICOM-CONSTRUCT\_TS1} &\tiny{FFAS03\_TS1} \\
\tiny{GDT\_TS = 0.53} &\tiny{GDT\_TS = 0.67} &\tiny{GDT\_TS = 0.78} &\tiny{GDT\_TS = 0.94} \\
\begin{minipage}{\linewidth}\includegraphics[width=\linewidth]{GradCAMOutput/T0766/T0766_RBO_Aleph_TS3.png}\end{minipage}&\begin{minipage}{\linewidth}\includegraphics[width=\linewidth]{GradCAMOutput/T0766/T0766_TASSER-VMT_TS2.png}\end{minipage}&\begin{minipage}{\linewidth}\includegraphics[width=\linewidth]{GradCAMOutput/T0766/T0766_MULTICOM-CONSTRUCT_TS1.png}\end{minipage}&\begin{minipage}{\linewidth}\includegraphics[width=\linewidth]{GradCAMOutput/T0766/T0766_FFAS03_TS1.png}\end{minipage}\\

\end{tabularx}
%}
\hskip\headheight}
\end{center}	
	\begin{center}
	\makebox[10pt][c]{
	\hskip-\footskip
	\begin{tabularx}{0.95\paperwidth}{X*{4}{p{5.0cm}}}
					\hline
\tiny{T0767} &\tiny{T0767} &\tiny{T0767} &\tiny{T0767} \\
\tiny{BioShell-server\_TS3} &\tiny{MULTICOM-CONSTRUCT\_TS1} &\tiny{Pcons-net\_TS1} &\tiny{RBO\_Aleph\_TS3} \\
\tiny{GDT\_TS = 0.08} &\tiny{GDT\_TS = 0.09} &\tiny{GDT\_TS = 0.12} &\tiny{GDT\_TS = 0.15} \\
\begin{minipage}{\linewidth}\includegraphics[width=\linewidth]{GradCAMOutput/T0767/T0767_BioShell-server_TS3.png}\end{minipage}&\begin{minipage}{\linewidth}\includegraphics[width=\linewidth]{GradCAMOutput/T0767/T0767_MULTICOM-CONSTRUCT_TS1.png}\end{minipage}&\begin{minipage}{\linewidth}\includegraphics[width=\linewidth]{GradCAMOutput/T0767/T0767_Pcons-net_TS1.png}\end{minipage}&\begin{minipage}{\linewidth}\includegraphics[width=\linewidth]{GradCAMOutput/T0767/T0767_RBO_Aleph_TS3.png}\end{minipage}\\
\hline
\tiny{T0768} &\tiny{T0768} &\tiny{T0768} &\tiny{T0768} \\
\tiny{Distill\_TS3} &\tiny{FFAS03\_TS1} &\tiny{FALCON\_MANUAL\_TS2} &\tiny{nns\_TS5} \\
\tiny{GDT\_TS = 0.37} &\tiny{GDT\_TS = 0.49} &\tiny{GDT\_TS = 0.65} &\tiny{GDT\_TS = 0.68} \\
\begin{minipage}{\linewidth}\includegraphics[width=\linewidth]{GradCAMOutput/T0768/T0768_Distill_TS3.png}\end{minipage}&\begin{minipage}{\linewidth}\includegraphics[width=\linewidth]{GradCAMOutput/T0768/T0768_FFAS03_TS1.png}\end{minipage}&\begin{minipage}{\linewidth}\includegraphics[width=\linewidth]{GradCAMOutput/T0768/T0768_FALCON_MANUAL_TS2.png}\end{minipage}&\begin{minipage}{\linewidth}\includegraphics[width=\linewidth]{GradCAMOutput/T0768/T0768_nns_TS5.png}\end{minipage}\\
\hline
\tiny{T0769} &\tiny{T0769} &\tiny{T0769} &\tiny{T0769} \\
\tiny{MUFOLD-Server\_TS4} &\tiny{MULTICOM-CONSTRUCT\_TS1} &\tiny{RBO\_Aleph\_TS3} &\tiny{BAKER-ROSETTASERVER\_TS4} \\
\tiny{GDT\_TS = 0.44} &\tiny{GDT\_TS = 0.48} &\tiny{GDT\_TS = 0.66} &\tiny{GDT\_TS = 0.68} \\
\begin{minipage}{\linewidth}\includegraphics[width=\linewidth]{GradCAMOutput/T0769/T0769_MUFOLD-Server_TS4.png}\end{minipage}&\begin{minipage}{\linewidth}\includegraphics[width=\linewidth]{GradCAMOutput/T0769/T0769_MULTICOM-CONSTRUCT_TS1.png}\end{minipage}&\begin{minipage}{\linewidth}\includegraphics[width=\linewidth]{GradCAMOutput/T0769/T0769_RBO_Aleph_TS3.png}\end{minipage}&\begin{minipage}{\linewidth}\includegraphics[width=\linewidth]{GradCAMOutput/T0769/T0769_BAKER-ROSETTASERVER_TS4.png}\end{minipage}\\
\hline
\tiny{T0770} &\tiny{T0770} &\tiny{T0770} &\tiny{T0770} \\
\tiny{MUFOLD-Server\_TS4} &\tiny{Distill\_TS3} &\tiny{RBO\_Aleph\_TS3} &\tiny{RBO\_Aleph\_TS4} \\
\tiny{GDT\_TS = 0.48} &\tiny{GDT\_TS = 0.55} &\tiny{GDT\_TS = 0.60} &\tiny{GDT\_TS = 0.62} \\
\begin{minipage}{\linewidth}\includegraphics[width=\linewidth]{GradCAMOutput/T0770/T0770_MUFOLD-Server_TS4.png}\end{minipage}&\begin{minipage}{\linewidth}\includegraphics[width=\linewidth]{GradCAMOutput/T0770/T0770_Distill_TS3.png}\end{minipage}&\begin{minipage}{\linewidth}\includegraphics[width=\linewidth]{GradCAMOutput/T0770/T0770_RBO_Aleph_TS3.png}\end{minipage}&\begin{minipage}{\linewidth}\includegraphics[width=\linewidth]{GradCAMOutput/T0770/T0770_RBO_Aleph_TS4.png}\end{minipage}\\

\end{tabularx}
%}
\hskip\headheight}
\end{center}	
	\begin{center}
	\makebox[10pt][c]{
	\hskip-\footskip
	\begin{tabularx}{0.95\paperwidth}{X*{4}{p{5.0cm}}}
					\hline
\tiny{T0771} &\tiny{T0771} &\tiny{T0771} &\tiny{T0771} \\
\tiny{MULTICOM-CONSTRUCT\_TS1} &\tiny{RBO\_Aleph\_TS3} &\tiny{RBO\_Aleph\_TS4} &\tiny{BAKER-ROSETTASERVER\_TS5} \\
\tiny{GDT\_TS = 0.10} &\tiny{GDT\_TS = 0.14} &\tiny{GDT\_TS = 0.15} &\tiny{GDT\_TS = 0.21} \\
\begin{minipage}{\linewidth}\includegraphics[width=\linewidth]{GradCAMOutput/T0771/T0771_MULTICOM-CONSTRUCT_TS1.png}\end{minipage}&\begin{minipage}{\linewidth}\includegraphics[width=\linewidth]{GradCAMOutput/T0771/T0771_RBO_Aleph_TS3.png}\end{minipage}&\begin{minipage}{\linewidth}\includegraphics[width=\linewidth]{GradCAMOutput/T0771/T0771_RBO_Aleph_TS4.png}\end{minipage}&\begin{minipage}{\linewidth}\includegraphics[width=\linewidth]{GradCAMOutput/T0771/T0771_BAKER-ROSETTASERVER_TS5.png}\end{minipage}\\
\hline
\tiny{T0772} &\tiny{T0772} &\tiny{T0772} &\tiny{T0772} \\
\tiny{Distill\_TS3} &\tiny{RBO\_Aleph\_TS3} &\tiny{BAKER-ROSETTASERVER\_TS4} &\tiny{FFAS03\_TS1} \\
\tiny{GDT\_TS = 0.45} &\tiny{GDT\_TS = 0.51} &\tiny{GDT\_TS = 0.56} &\tiny{GDT\_TS = 0.59} \\
\begin{minipage}{\linewidth}\includegraphics[width=\linewidth]{GradCAMOutput/T0772/T0772_Distill_TS3.png}\end{minipage}&\begin{minipage}{\linewidth}\includegraphics[width=\linewidth]{GradCAMOutput/T0772/T0772_RBO_Aleph_TS3.png}\end{minipage}&\begin{minipage}{\linewidth}\includegraphics[width=\linewidth]{GradCAMOutput/T0772/T0772_BAKER-ROSETTASERVER_TS4.png}\end{minipage}&\begin{minipage}{\linewidth}\includegraphics[width=\linewidth]{GradCAMOutput/T0772/T0772_FFAS03_TS1.png}\end{minipage}\\
\hline
\tiny{T0773} &\tiny{T0773} &\tiny{T0773} &\tiny{T0773} \\
\tiny{eThread\_TS1} &\tiny{MULTICOM-CLUSTER\_TS2} &\tiny{MULTICOM-CONSTRUCT\_TS1} &\tiny{RBO\_Aleph\_TS3} \\
\tiny{GDT\_TS = 0.44} &\tiny{GDT\_TS = 0.57} &\tiny{GDT\_TS = 0.61} &\tiny{GDT\_TS = 0.82} \\
\begin{minipage}{\linewidth}\includegraphics[width=\linewidth]{GradCAMOutput/T0773/T0773_eThread_TS1.png}\end{minipage}&\begin{minipage}{\linewidth}\includegraphics[width=\linewidth]{GradCAMOutput/T0773/T0773_MULTICOM-CLUSTER_TS2.png}\end{minipage}&\begin{minipage}{\linewidth}\includegraphics[width=\linewidth]{GradCAMOutput/T0773/T0773_MULTICOM-CONSTRUCT_TS1.png}\end{minipage}&\begin{minipage}{\linewidth}\includegraphics[width=\linewidth]{GradCAMOutput/T0773/T0773_RBO_Aleph_TS3.png}\end{minipage}\\
\hline
\tiny{T0774} &\tiny{T0774} &\tiny{T0774} &\tiny{T0774} \\
\tiny{Seok-server\_TS4} &\tiny{RBO\_Aleph\_TS3} &\tiny{FFAS03\_TS1} &\tiny{Atome2\_CBS\_TS3} \\
\tiny{GDT\_TS = 0.20} &\tiny{GDT\_TS = 0.27} &\tiny{GDT\_TS = 0.33} &\tiny{GDT\_TS = 0.43} \\
\begin{minipage}{\linewidth}\includegraphics[width=\linewidth]{GradCAMOutput/T0774/T0774_Seok-server_TS4.png}\end{minipage}&\begin{minipage}{\linewidth}\includegraphics[width=\linewidth]{GradCAMOutput/T0774/T0774_RBO_Aleph_TS3.png}\end{minipage}&\begin{minipage}{\linewidth}\includegraphics[width=\linewidth]{GradCAMOutput/T0774/T0774_FFAS03_TS1.png}\end{minipage}&\begin{minipage}{\linewidth}\includegraphics[width=\linewidth]{GradCAMOutput/T0774/T0774_Atome2_CBS_TS3.png}\end{minipage}\\

\end{tabularx}
%}
\hskip\headheight}
\end{center}	
	\begin{center}
	\makebox[10pt][c]{
	\hskip-\footskip
	\begin{tabularx}{0.95\paperwidth}{X*{4}{p{5.0cm}}}
					\hline
\tiny{T0776} &\tiny{T0776} &\tiny{T0776} &\tiny{T0776} \\
\tiny{Distill\_TS3} &\tiny{RBO\_Aleph\_TS3} &\tiny{MUFOLD-Server\_TS4} &\tiny{FFAS03\_TS1} \\
\tiny{GDT\_TS = 0.65} &\tiny{GDT\_TS = 0.68} &\tiny{GDT\_TS = 0.75} &\tiny{GDT\_TS = 0.82} \\
\begin{minipage}{\linewidth}\includegraphics[width=\linewidth]{GradCAMOutput/T0776/T0776_Distill_TS3.png}\end{minipage}&\begin{minipage}{\linewidth}\includegraphics[width=\linewidth]{GradCAMOutput/T0776/T0776_RBO_Aleph_TS3.png}\end{minipage}&\begin{minipage}{\linewidth}\includegraphics[width=\linewidth]{GradCAMOutput/T0776/T0776_MUFOLD-Server_TS4.png}\end{minipage}&\begin{minipage}{\linewidth}\includegraphics[width=\linewidth]{GradCAMOutput/T0776/T0776_FFAS03_TS1.png}\end{minipage}\\
\hline
\tiny{T0777} &\tiny{T0777} &\tiny{T0777} &\tiny{T0777} \\
\tiny{Atome2\_CBS\_TS3} &\tiny{MULTICOM-CONSTRUCT\_TS1} &\tiny{RBO\_Aleph\_TS3} &\tiny{RBO\_Aleph\_TS4} \\
\tiny{GDT\_TS = 0.08} &\tiny{GDT\_TS = 0.10} &\tiny{GDT\_TS = 0.11} &\tiny{GDT\_TS = 0.13} \\
\begin{minipage}{\linewidth}\includegraphics[width=\linewidth]{GradCAMOutput/T0777/T0777_Atome2_CBS_TS3.png}\end{minipage}&\begin{minipage}{\linewidth}\includegraphics[width=\linewidth]{GradCAMOutput/T0777/T0777_MULTICOM-CONSTRUCT_TS1.png}\end{minipage}&\begin{minipage}{\linewidth}\includegraphics[width=\linewidth]{GradCAMOutput/T0777/T0777_RBO_Aleph_TS3.png}\end{minipage}&\begin{minipage}{\linewidth}\includegraphics[width=\linewidth]{GradCAMOutput/T0777/T0777_RBO_Aleph_TS4.png}\end{minipage}\\
\hline
\tiny{T0780} &\tiny{T0780} &\tiny{T0780} &\tiny{T0780} \\
\tiny{RBO\_Aleph\_TS3} &\tiny{slbio\_TS3} &\tiny{BioShell-server\_TS3} &\tiny{MULTICOM-CONSTRUCT\_TS1} \\
\tiny{GDT\_TS = 0.09} &\tiny{GDT\_TS = 0.17} &\tiny{GDT\_TS = 0.24} &\tiny{GDT\_TS = 0.27} \\
\begin{minipage}{\linewidth}\includegraphics[width=\linewidth]{GradCAMOutput/T0780/T0780_RBO_Aleph_TS3.png}\end{minipage}&\begin{minipage}{\linewidth}\includegraphics[width=\linewidth]{GradCAMOutput/T0780/T0780_slbio_TS3.png}\end{minipage}&\begin{minipage}{\linewidth}\includegraphics[width=\linewidth]{GradCAMOutput/T0780/T0780_BioShell-server_TS3.png}\end{minipage}&\begin{minipage}{\linewidth}\includegraphics[width=\linewidth]{GradCAMOutput/T0780/T0780_MULTICOM-CONSTRUCT_TS1.png}\end{minipage}\\
\hline
\tiny{T0781} &\tiny{T0781} &\tiny{T0781} &\tiny{T0781} \\
\tiny{Distill\_TS3} &\tiny{RBO\_Aleph\_TS3} &\tiny{MULTICOM-CLUSTER\_TS2} &\tiny{RaptorX\_TS1} \\
\tiny{GDT\_TS = 0.07} &\tiny{GDT\_TS = 0.09} &\tiny{GDT\_TS = 0.12} &\tiny{GDT\_TS = 0.15} \\
\begin{minipage}{\linewidth}\includegraphics[width=\linewidth]{GradCAMOutput/T0781/T0781_Distill_TS3.png}\end{minipage}&\begin{minipage}{\linewidth}\includegraphics[width=\linewidth]{GradCAMOutput/T0781/T0781_RBO_Aleph_TS3.png}\end{minipage}&\begin{minipage}{\linewidth}\includegraphics[width=\linewidth]{GradCAMOutput/T0781/T0781_MULTICOM-CLUSTER_TS2.png}\end{minipage}&\begin{minipage}{\linewidth}\includegraphics[width=\linewidth]{GradCAMOutput/T0781/T0781_RaptorX_TS1.png}\end{minipage}\\

\end{tabularx}
%}
\hskip\headheight}
\end{center}	
	\begin{center}
	\makebox[10pt][c]{
	\hskip-\footskip
	\begin{tabularx}{0.95\paperwidth}{X*{4}{p{5.0cm}}}
					\hline
\tiny{T0782} &\tiny{T0782} &\tiny{T0782} &\tiny{T0782} \\
\tiny{RBO\_Aleph\_TS3} &\tiny{FFAS03\_TS1} &\tiny{MULTICOM-CONSTRUCT\_TS1} &\tiny{BAKER-ROSETTASERVER\_TS4} \\
\tiny{GDT\_TS = 0.26} &\tiny{GDT\_TS = 0.43} &\tiny{GDT\_TS = 0.48} &\tiny{GDT\_TS = 0.60} \\
\begin{minipage}{\linewidth}\includegraphics[width=\linewidth]{GradCAMOutput/T0782/T0782_RBO_Aleph_TS3.png}\end{minipage}&\begin{minipage}{\linewidth}\includegraphics[width=\linewidth]{GradCAMOutput/T0782/T0782_FFAS03_TS1.png}\end{minipage}&\begin{minipage}{\linewidth}\includegraphics[width=\linewidth]{GradCAMOutput/T0782/T0782_MULTICOM-CONSTRUCT_TS1.png}\end{minipage}&\begin{minipage}{\linewidth}\includegraphics[width=\linewidth]{GradCAMOutput/T0782/T0782_BAKER-ROSETTASERVER_TS4.png}\end{minipage}\\
\hline
\tiny{T0783} &\tiny{T0783} &\tiny{T0783} &\tiny{T0783} \\
\tiny{MUFOLD-Server\_TS4} &\tiny{FFAS03\_TS1} &\tiny{BAKER-ROSETTASERVER\_TS4} &\tiny{RBO\_Aleph\_TS3} \\
\tiny{GDT\_TS = 0.30} &\tiny{GDT\_TS = 0.36} &\tiny{GDT\_TS = 0.40} &\tiny{GDT\_TS = 0.43} \\
\begin{minipage}{\linewidth}\includegraphics[width=\linewidth]{GradCAMOutput/T0783/T0783_MUFOLD-Server_TS4.png}\end{minipage}&\begin{minipage}{\linewidth}\includegraphics[width=\linewidth]{GradCAMOutput/T0783/T0783_FFAS03_TS1.png}\end{minipage}&\begin{minipage}{\linewidth}\includegraphics[width=\linewidth]{GradCAMOutput/T0783/T0783_BAKER-ROSETTASERVER_TS4.png}\end{minipage}&\begin{minipage}{\linewidth}\includegraphics[width=\linewidth]{GradCAMOutput/T0783/T0783_RBO_Aleph_TS3.png}\end{minipage}\\
\hline
\tiny{T0784} &\tiny{T0784} &\tiny{T0784} &\tiny{T0784} \\
\tiny{eThread\_TS5} &\tiny{BioShell-server\_TS3} &\tiny{MULTICOM-CONSTRUCT\_TS1} &\tiny{FFAS03\_TS1} \\
\tiny{GDT\_TS = 0.50} &\tiny{GDT\_TS = 0.67} &\tiny{GDT\_TS = 0.71} &\tiny{GDT\_TS = 0.87} \\
\begin{minipage}{\linewidth}\includegraphics[width=\linewidth]{GradCAMOutput/T0784/T0784_eThread_TS5.png}\end{minipage}&\begin{minipage}{\linewidth}\includegraphics[width=\linewidth]{GradCAMOutput/T0784/T0784_BioShell-server_TS3.png}\end{minipage}&\begin{minipage}{\linewidth}\includegraphics[width=\linewidth]{GradCAMOutput/T0784/T0784_MULTICOM-CONSTRUCT_TS1.png}\end{minipage}&\begin{minipage}{\linewidth}\includegraphics[width=\linewidth]{GradCAMOutput/T0784/T0784_FFAS03_TS1.png}\end{minipage}\\
\hline
\tiny{T0785} &\tiny{T0785} &\tiny{T0785} &\tiny{T0785} \\
\tiny{FFAS03\_TS1} &\tiny{nns\_TS5} &\tiny{RBO\_Aleph\_TS3} &\tiny{BAKER-ROSETTASERVER\_TS2} \\
\tiny{GDT\_TS = 0.14} &\tiny{GDT\_TS = 0.17} &\tiny{GDT\_TS = 0.21} &\tiny{GDT\_TS = 0.25} \\
\begin{minipage}{\linewidth}\includegraphics[width=\linewidth]{GradCAMOutput/T0785/T0785_FFAS03_TS1.png}\end{minipage}&\begin{minipage}{\linewidth}\includegraphics[width=\linewidth]{GradCAMOutput/T0785/T0785_nns_TS5.png}\end{minipage}&\begin{minipage}{\linewidth}\includegraphics[width=\linewidth]{GradCAMOutput/T0785/T0785_RBO_Aleph_TS3.png}\end{minipage}&\begin{minipage}{\linewidth}\includegraphics[width=\linewidth]{GradCAMOutput/T0785/T0785_BAKER-ROSETTASERVER_TS2.png}\end{minipage}\\

\end{tabularx}
%}
\hskip\headheight}
\end{center}	
	\begin{center}
	\makebox[10pt][c]{
	\hskip-\footskip
	\begin{tabularx}{0.95\paperwidth}{X*{4}{p{5.0cm}}}
					\hline
\tiny{T0786} &\tiny{T0786} &\tiny{T0786} &\tiny{T0786} \\
\tiny{eThread\_TS2} &\tiny{FFAS03\_TS1} &\tiny{RBO\_Aleph\_TS3} &\tiny{MULTICOM-CONSTRUCT\_TS1} \\
\tiny{GDT\_TS = 0.32} &\tiny{GDT\_TS = 0.43} &\tiny{GDT\_TS = 0.47} &\tiny{GDT\_TS = 0.53} \\
\begin{minipage}{\linewidth}\includegraphics[width=\linewidth]{GradCAMOutput/T0786/T0786_eThread_TS2.png}\end{minipage}&\begin{minipage}{\linewidth}\includegraphics[width=\linewidth]{GradCAMOutput/T0786/T0786_FFAS03_TS1.png}\end{minipage}&\begin{minipage}{\linewidth}\includegraphics[width=\linewidth]{GradCAMOutput/T0786/T0786_RBO_Aleph_TS3.png}\end{minipage}&\begin{minipage}{\linewidth}\includegraphics[width=\linewidth]{GradCAMOutput/T0786/T0786_MULTICOM-CONSTRUCT_TS1.png}\end{minipage}\\
\hline
\tiny{T0787} &\tiny{T0787} &\tiny{T0787} &\tiny{T0787} \\
\tiny{MUFOLD-Server\_TS4} &\tiny{RBO\_Aleph\_TS3} &\tiny{RBO\_Aleph\_TS4} &\tiny{FFAS03\_TS1} \\
\tiny{GDT\_TS = 0.16} &\tiny{GDT\_TS = 0.18} &\tiny{GDT\_TS = 0.19} &\tiny{GDT\_TS = 0.20} \\
\begin{minipage}{\linewidth}\includegraphics[width=\linewidth]{GradCAMOutput/T0787/T0787_MUFOLD-Server_TS4.png}\end{minipage}&\begin{minipage}{\linewidth}\includegraphics[width=\linewidth]{GradCAMOutput/T0787/T0787_RBO_Aleph_TS3.png}\end{minipage}&\begin{minipage}{\linewidth}\includegraphics[width=\linewidth]{GradCAMOutput/T0787/T0787_RBO_Aleph_TS4.png}\end{minipage}&\begin{minipage}{\linewidth}\includegraphics[width=\linewidth]{GradCAMOutput/T0787/T0787_FFAS03_TS1.png}\end{minipage}\\
\hline
\tiny{T0788} &\tiny{T0788} &\tiny{T0788} &\tiny{T0788} \\
\tiny{RBO\_Aleph\_TS4} &\tiny{MULTICOM-CLUSTER\_TS2} &\tiny{FFAS03\_TS1} &\tiny{QUARK\_TS3} \\
\tiny{GDT\_TS = 0.55} &\tiny{GDT\_TS = 0.60} &\tiny{GDT\_TS = 0.71} &\tiny{GDT\_TS = 0.74} \\
\begin{minipage}{\linewidth}\includegraphics[width=\linewidth]{GradCAMOutput/T0788/T0788_RBO_Aleph_TS4.png}\end{minipage}&\begin{minipage}{\linewidth}\includegraphics[width=\linewidth]{GradCAMOutput/T0788/T0788_MULTICOM-CLUSTER_TS2.png}\end{minipage}&\begin{minipage}{\linewidth}\includegraphics[width=\linewidth]{GradCAMOutput/T0788/T0788_FFAS03_TS1.png}\end{minipage}&\begin{minipage}{\linewidth}\includegraphics[width=\linewidth]{GradCAMOutput/T0788/T0788_QUARK_TS3.png}\end{minipage}\\
\hline
\tiny{T0789} &\tiny{T0789} &\tiny{T0789} &\tiny{T0789} \\
\tiny{RBO\_Aleph\_TS3} &\tiny{MULTICOM-CONSTRUCT\_TS1} &\tiny{MULTICOM-NOVEL\_TS4} &\tiny{BAKER-ROSETTASERVER\_TS2} \\
\tiny{GDT\_TS = 0.09} &\tiny{GDT\_TS = 0.11} &\tiny{GDT\_TS = 0.13} &\tiny{GDT\_TS = 0.18} \\
\begin{minipage}{\linewidth}\includegraphics[width=\linewidth]{GradCAMOutput/T0789/T0789_RBO_Aleph_TS3.png}\end{minipage}&\begin{minipage}{\linewidth}\includegraphics[width=\linewidth]{GradCAMOutput/T0789/T0789_MULTICOM-CONSTRUCT_TS1.png}\end{minipage}&\begin{minipage}{\linewidth}\includegraphics[width=\linewidth]{GradCAMOutput/T0789/T0789_MULTICOM-NOVEL_TS4.png}\end{minipage}&\begin{minipage}{\linewidth}\includegraphics[width=\linewidth]{GradCAMOutput/T0789/T0789_BAKER-ROSETTASERVER_TS2.png}\end{minipage}\\

\end{tabularx}
%}
\hskip\headheight}
\end{center}	
	\begin{center}
	\makebox[10pt][c]{
	\hskip-\footskip
	\begin{tabularx}{0.95\paperwidth}{X*{4}{p{5.0cm}}}
					\hline
\tiny{T0790} &\tiny{T0790} &\tiny{T0790} &\tiny{T0790} \\
\tiny{MULTICOM-CONSTRUCT\_TS1} &\tiny{Distill\_TS3} &\tiny{BAKER-ROSETTASERVER\_TS2} &\tiny{BAKER-ROSETTASERVER\_TS3} \\
\tiny{GDT\_TS = 0.11} &\tiny{GDT\_TS = 0.12} &\tiny{GDT\_TS = 0.16} &\tiny{GDT\_TS = 0.22} \\
\begin{minipage}{\linewidth}\includegraphics[width=\linewidth]{GradCAMOutput/T0790/T0790_MULTICOM-CONSTRUCT_TS1.png}\end{minipage}&\begin{minipage}{\linewidth}\includegraphics[width=\linewidth]{GradCAMOutput/T0790/T0790_Distill_TS3.png}\end{minipage}&\begin{minipage}{\linewidth}\includegraphics[width=\linewidth]{GradCAMOutput/T0790/T0790_BAKER-ROSETTASERVER_TS2.png}\end{minipage}&\begin{minipage}{\linewidth}\includegraphics[width=\linewidth]{GradCAMOutput/T0790/T0790_BAKER-ROSETTASERVER_TS3.png}\end{minipage}\\
\hline
\tiny{T0792} &\tiny{T0792} &\tiny{T0792} &\tiny{T0792} \\
\tiny{BioShell-server\_TS3} &\tiny{MULTICOM-CONSTRUCT\_TS1} &\tiny{FFAS03\_TS1} &\tiny{BAKER-ROSETTASERVER\_TS4} \\
\tiny{GDT\_TS = 0.56} &\tiny{GDT\_TS = 0.63} &\tiny{GDT\_TS = 0.72} &\tiny{GDT\_TS = 0.79} \\
\begin{minipage}{\linewidth}\includegraphics[width=\linewidth]{GradCAMOutput/T0792/T0792_BioShell-server_TS3.png}\end{minipage}&\begin{minipage}{\linewidth}\includegraphics[width=\linewidth]{GradCAMOutput/T0792/T0792_MULTICOM-CONSTRUCT_TS1.png}\end{minipage}&\begin{minipage}{\linewidth}\includegraphics[width=\linewidth]{GradCAMOutput/T0792/T0792_FFAS03_TS1.png}\end{minipage}&\begin{minipage}{\linewidth}\includegraphics[width=\linewidth]{GradCAMOutput/T0792/T0792_BAKER-ROSETTASERVER_TS4.png}\end{minipage}\\
\hline
\tiny{T0794} &\tiny{T0794} &\tiny{T0794} &\tiny{T0794} \\
\tiny{BioShell-server\_TS5} &\tiny{RBO\_Aleph\_TS3} &\tiny{FALCON\_MANUAL\_TS2} &\tiny{MULTICOM-CONSTRUCT\_TS1} \\
\tiny{GDT\_TS = 0.29} &\tiny{GDT\_TS = 0.32} &\tiny{GDT\_TS = 0.38} &\tiny{GDT\_TS = 0.41} \\
\begin{minipage}{\linewidth}\includegraphics[width=\linewidth]{GradCAMOutput/T0794/T0794_BioShell-server_TS5.png}\end{minipage}&\begin{minipage}{\linewidth}\includegraphics[width=\linewidth]{GradCAMOutput/T0794/T0794_RBO_Aleph_TS3.png}\end{minipage}&\begin{minipage}{\linewidth}\includegraphics[width=\linewidth]{GradCAMOutput/T0794/T0794_FALCON_MANUAL_TS2.png}\end{minipage}&\begin{minipage}{\linewidth}\includegraphics[width=\linewidth]{GradCAMOutput/T0794/T0794_MULTICOM-CONSTRUCT_TS1.png}\end{minipage}\\
\hline
\tiny{T0796} &\tiny{T0796} &\tiny{T0796} &\tiny{T0796} \\
\tiny{FALCON\_MANUAL\_TS2} &\tiny{RBO\_Aleph\_TS3} &\tiny{MULTICOM-CONSTRUCT\_TS1} &\tiny{FFAS03\_TS1} \\
\tiny{GDT\_TS = 0.32} &\tiny{GDT\_TS = 0.39} &\tiny{GDT\_TS = 0.41} &\tiny{GDT\_TS = 0.53} \\
\begin{minipage}{\linewidth}\includegraphics[width=\linewidth]{GradCAMOutput/T0796/T0796_FALCON_MANUAL_TS2.png}\end{minipage}&\begin{minipage}{\linewidth}\includegraphics[width=\linewidth]{GradCAMOutput/T0796/T0796_RBO_Aleph_TS3.png}\end{minipage}&\begin{minipage}{\linewidth}\includegraphics[width=\linewidth]{GradCAMOutput/T0796/T0796_MULTICOM-CONSTRUCT_TS1.png}\end{minipage}&\begin{minipage}{\linewidth}\includegraphics[width=\linewidth]{GradCAMOutput/T0796/T0796_FFAS03_TS1.png}\end{minipage}\\

\end{tabularx}
%}
\hskip\headheight}
\end{center}	
	\begin{center}
	\makebox[10pt][c]{
	\hskip-\footskip
	\begin{tabularx}{0.95\paperwidth}{X*{4}{p{5.0cm}}}
					\hline
\tiny{T0797} &\tiny{T0797} &\tiny{T0797} &\tiny{T0797} \\
\tiny{BAKER-ROSETTASERVER\_TS2} &\tiny{Alpha-Gelly-Server\_TS2} &\tiny{FFAS03\_TS1} &\tiny{MULTICOM-CONSTRUCT\_TS1} \\
\tiny{GDT\_TS = 0.52} &\tiny{GDT\_TS = 0.66} &\tiny{GDT\_TS = 0.73} &\tiny{GDT\_TS = 0.80} \\
\begin{minipage}{\linewidth}\includegraphics[width=\linewidth]{GradCAMOutput/T0797/T0797_BAKER-ROSETTASERVER_TS2.png}\end{minipage}&\begin{minipage}{\linewidth}\includegraphics[width=\linewidth]{GradCAMOutput/T0797/T0797_Alpha-Gelly-Server_TS2.png}\end{minipage}&\begin{minipage}{\linewidth}\includegraphics[width=\linewidth]{GradCAMOutput/T0797/T0797_FFAS03_TS1.png}\end{minipage}&\begin{minipage}{\linewidth}\includegraphics[width=\linewidth]{GradCAMOutput/T0797/T0797_MULTICOM-CONSTRUCT_TS1.png}\end{minipage}\\
\hline
\tiny{T0798} &\tiny{T0798} &\tiny{T0798} &\tiny{T0798} \\
\tiny{BioShell-server\_TS3} &\tiny{MULTICOM-CONSTRUCT\_TS1} &\tiny{Atome2\_CBS\_TS3} &\tiny{FFAS03\_TS1} \\
\tiny{GDT\_TS = 0.79} &\tiny{GDT\_TS = 0.80} &\tiny{GDT\_TS = 0.87} &\tiny{GDT\_TS = 0.89} \\
\begin{minipage}{\linewidth}\includegraphics[width=\linewidth]{GradCAMOutput/T0798/T0798_BioShell-server_TS3.png}\end{minipage}&\begin{minipage}{\linewidth}\includegraphics[width=\linewidth]{GradCAMOutput/T0798/T0798_MULTICOM-CONSTRUCT_TS1.png}\end{minipage}&\begin{minipage}{\linewidth}\includegraphics[width=\linewidth]{GradCAMOutput/T0798/T0798_Atome2_CBS_TS3.png}\end{minipage}&\begin{minipage}{\linewidth}\includegraphics[width=\linewidth]{GradCAMOutput/T0798/T0798_FFAS03_TS1.png}\end{minipage}\\
\hline
\tiny{T0800} &\tiny{T0800} &\tiny{T0800} &\tiny{T0800} \\
\tiny{MUFOLD-Server\_TS4} &\tiny{eThread\_TS1} &\tiny{RBO\_Aleph\_TS3} &\tiny{BAKER-ROSETTASERVER\_TS4} \\
\tiny{GDT\_TS = 0.16} &\tiny{GDT\_TS = 0.24} &\tiny{GDT\_TS = 0.30} &\tiny{GDT\_TS = 0.37} \\
\begin{minipage}{\linewidth}\includegraphics[width=\linewidth]{GradCAMOutput/T0800/T0800_MUFOLD-Server_TS4.png}\end{minipage}&\begin{minipage}{\linewidth}\includegraphics[width=\linewidth]{GradCAMOutput/T0800/T0800_eThread_TS1.png}\end{minipage}&\begin{minipage}{\linewidth}\includegraphics[width=\linewidth]{GradCAMOutput/T0800/T0800_RBO_Aleph_TS3.png}\end{minipage}&\begin{minipage}{\linewidth}\includegraphics[width=\linewidth]{GradCAMOutput/T0800/T0800_BAKER-ROSETTASERVER_TS4.png}\end{minipage}\\
\hline
\tiny{T0801} &\tiny{T0801} &\tiny{T0801} &\tiny{T0801} \\
\tiny{eThread\_TS2} &\tiny{FFAS03\_TS1} &\tiny{MUFOLD-Server\_TS4} &\tiny{MULTICOM-CONSTRUCT\_TS1} \\
\tiny{GDT\_TS = 0.72} &\tiny{GDT\_TS = 0.77} &\tiny{GDT\_TS = 0.80} &\tiny{GDT\_TS = 0.82} \\
\begin{minipage}{\linewidth}\includegraphics[width=\linewidth]{GradCAMOutput/T0801/T0801_eThread_TS2.png}\end{minipage}&\begin{minipage}{\linewidth}\includegraphics[width=\linewidth]{GradCAMOutput/T0801/T0801_FFAS03_TS1.png}\end{minipage}&\begin{minipage}{\linewidth}\includegraphics[width=\linewidth]{GradCAMOutput/T0801/T0801_MUFOLD-Server_TS4.png}\end{minipage}&\begin{minipage}{\linewidth}\includegraphics[width=\linewidth]{GradCAMOutput/T0801/T0801_MULTICOM-CONSTRUCT_TS1.png}\end{minipage}\\

\end{tabularx}
%}
\hskip\headheight}
\end{center}	
	\begin{center}
	\makebox[10pt][c]{
	\hskip-\footskip
	\begin{tabularx}{0.95\paperwidth}{X*{4}{p{5.0cm}}}
					\hline
\tiny{T0803} &\tiny{T0803} &\tiny{T0803} &\tiny{T0803} \\
\tiny{Distill\_TS3} &\tiny{FALCON\_MANUAL\_TS2} &\tiny{RBO\_Aleph\_TS3} &\tiny{MULTICOM-CONSTRUCT\_TS1} \\
\tiny{GDT\_TS = 0.22} &\tiny{GDT\_TS = 0.33} &\tiny{GDT\_TS = 0.40} &\tiny{GDT\_TS = 0.44} \\
\begin{minipage}{\linewidth}\includegraphics[width=\linewidth]{GradCAMOutput/T0803/T0803_Distill_TS3.png}\end{minipage}&\begin{minipage}{\linewidth}\includegraphics[width=\linewidth]{GradCAMOutput/T0803/T0803_FALCON_MANUAL_TS2.png}\end{minipage}&\begin{minipage}{\linewidth}\includegraphics[width=\linewidth]{GradCAMOutput/T0803/T0803_RBO_Aleph_TS3.png}\end{minipage}&\begin{minipage}{\linewidth}\includegraphics[width=\linewidth]{GradCAMOutput/T0803/T0803_MULTICOM-CONSTRUCT_TS1.png}\end{minipage}\\
\hline
\tiny{T0805} &\tiny{T0805} &\tiny{T0805} &\tiny{T0805} \\
\tiny{MUFOLD-Server\_TS4} &\tiny{Seok-server\_TS2} &\tiny{FFAS03\_TS1} &\tiny{RBO\_Aleph\_TS3} \\
\tiny{GDT\_TS = 0.60} &\tiny{GDT\_TS = 0.61} &\tiny{GDT\_TS = 0.65} &\tiny{GDT\_TS = 0.69} \\
\begin{minipage}{\linewidth}\includegraphics[width=\linewidth]{GradCAMOutput/T0805/T0805_MUFOLD-Server_TS4.png}\end{minipage}&\begin{minipage}{\linewidth}\includegraphics[width=\linewidth]{GradCAMOutput/T0805/T0805_Seok-server_TS2.png}\end{minipage}&\begin{minipage}{\linewidth}\includegraphics[width=\linewidth]{GradCAMOutput/T0805/T0805_FFAS03_TS1.png}\end{minipage}&\begin{minipage}{\linewidth}\includegraphics[width=\linewidth]{GradCAMOutput/T0805/T0805_RBO_Aleph_TS3.png}\end{minipage}\\
\hline
\tiny{T0806} &\tiny{T0806} &\tiny{T0806} &\tiny{T0806} \\
\tiny{MULTICOM-CONSTRUCT\_TS1} &\tiny{BAKER-ROSETTASERVER\_TS4} &\tiny{RBO\_Aleph\_TS3} &\tiny{BAKER-ROSETTASERVER\_TS2} \\
\tiny{GDT\_TS = 0.12} &\tiny{GDT\_TS = 0.13} &\tiny{GDT\_TS = 0.19} &\tiny{GDT\_TS = 0.24} \\
\begin{minipage}{\linewidth}\includegraphics[width=\linewidth]{GradCAMOutput/T0806/T0806_MULTICOM-CONSTRUCT_TS1.png}\end{minipage}&\begin{minipage}{\linewidth}\includegraphics[width=\linewidth]{GradCAMOutput/T0806/T0806_BAKER-ROSETTASERVER_TS4.png}\end{minipage}&\begin{minipage}{\linewidth}\includegraphics[width=\linewidth]{GradCAMOutput/T0806/T0806_RBO_Aleph_TS3.png}\end{minipage}&\begin{minipage}{\linewidth}\includegraphics[width=\linewidth]{GradCAMOutput/T0806/T0806_BAKER-ROSETTASERVER_TS2.png}\end{minipage}\\
\hline
\tiny{T0807} &\tiny{T0807} &\tiny{T0807} &\tiny{T0807} \\
\tiny{BioShell-server\_TS3} &\tiny{MUFOLD-Server\_TS4} &\tiny{BAKER-ROSETTASERVER\_TS4} &\tiny{RBO\_Aleph\_TS3} \\
\tiny{GDT\_TS = 0.69} &\tiny{GDT\_TS = 0.73} &\tiny{GDT\_TS = 0.78} &\tiny{GDT\_TS = 0.79} \\
\begin{minipage}{\linewidth}\includegraphics[width=\linewidth]{GradCAMOutput/T0807/T0807_BioShell-server_TS3.png}\end{minipage}&\begin{minipage}{\linewidth}\includegraphics[width=\linewidth]{GradCAMOutput/T0807/T0807_MUFOLD-Server_TS4.png}\end{minipage}&\begin{minipage}{\linewidth}\includegraphics[width=\linewidth]{GradCAMOutput/T0807/T0807_BAKER-ROSETTASERVER_TS4.png}\end{minipage}&\begin{minipage}{\linewidth}\includegraphics[width=\linewidth]{GradCAMOutput/T0807/T0807_RBO_Aleph_TS3.png}\end{minipage}\\

\end{tabularx}
%}
\hskip\headheight}
\end{center}	
	\begin{center}
	\makebox[10pt][c]{
	\hskip-\footskip
	\begin{tabularx}{0.95\paperwidth}{X*{4}{p{5.0cm}}}
					\hline
\tiny{T0808} &\tiny{T0808} &\tiny{T0808} &\tiny{T0808} \\
\tiny{MUFOLD-Server\_TS4} &\tiny{slbio\_TS3} &\tiny{RBO\_Aleph\_TS3} &\tiny{MULTICOM-CONSTRUCT\_TS1} \\
\tiny{GDT\_TS = 0.08} &\tiny{GDT\_TS = 0.12} &\tiny{GDT\_TS = 0.16} &\tiny{GDT\_TS = 0.18} \\
\begin{minipage}{\linewidth}\includegraphics[width=\linewidth]{GradCAMOutput/T0808/T0808_MUFOLD-Server_TS4.png}\end{minipage}&\begin{minipage}{\linewidth}\includegraphics[width=\linewidth]{GradCAMOutput/T0808/T0808_slbio_TS3.png}\end{minipage}&\begin{minipage}{\linewidth}\includegraphics[width=\linewidth]{GradCAMOutput/T0808/T0808_RBO_Aleph_TS3.png}\end{minipage}&\begin{minipage}{\linewidth}\includegraphics[width=\linewidth]{GradCAMOutput/T0808/T0808_MULTICOM-CONSTRUCT_TS1.png}\end{minipage}\\
\hline
\tiny{T0810} &\tiny{T0810} &\tiny{T0810} &\tiny{T0810} \\
\tiny{BioShell-server\_TS3} &\tiny{eThread\_TS1} &\tiny{MUFOLD-Server\_TS4} &\tiny{MULTICOM-CONSTRUCT\_TS1} \\
\tiny{GDT\_TS = 0.34} &\tiny{GDT\_TS = 0.37} &\tiny{GDT\_TS = 0.38} &\tiny{GDT\_TS = 0.40} \\
\begin{minipage}{\linewidth}\includegraphics[width=\linewidth]{GradCAMOutput/T0810/T0810_BioShell-server_TS3.png}\end{minipage}&\begin{minipage}{\linewidth}\includegraphics[width=\linewidth]{GradCAMOutput/T0810/T0810_eThread_TS1.png}\end{minipage}&\begin{minipage}{\linewidth}\includegraphics[width=\linewidth]{GradCAMOutput/T0810/T0810_MUFOLD-Server_TS4.png}\end{minipage}&\begin{minipage}{\linewidth}\includegraphics[width=\linewidth]{GradCAMOutput/T0810/T0810_MULTICOM-CONSTRUCT_TS1.png}\end{minipage}\\
\hline
\tiny{T0811} &\tiny{T0811} &\tiny{T0811} &\tiny{T0811} \\
\tiny{MUFOLD-Server\_TS4} &\tiny{BioShell-server\_TS3} &\tiny{Distill\_TS3} &\tiny{FFAS03\_TS1} \\
\tiny{GDT\_TS = 0.78} &\tiny{GDT\_TS = 0.80} &\tiny{GDT\_TS = 0.86} &\tiny{GDT\_TS = 0.88} \\
\begin{minipage}{\linewidth}\includegraphics[width=\linewidth]{GradCAMOutput/T0811/T0811_MUFOLD-Server_TS4.png}\end{minipage}&\begin{minipage}{\linewidth}\includegraphics[width=\linewidth]{GradCAMOutput/T0811/T0811_BioShell-server_TS3.png}\end{minipage}&\begin{minipage}{\linewidth}\includegraphics[width=\linewidth]{GradCAMOutput/T0811/T0811_Distill_TS3.png}\end{minipage}&\begin{minipage}{\linewidth}\includegraphics[width=\linewidth]{GradCAMOutput/T0811/T0811_FFAS03_TS1.png}\end{minipage}\\
\hline
\tiny{T0812} &\tiny{T0812} &\tiny{T0812} &\tiny{T0812} \\
\tiny{RBO\_Aleph\_TS3} &\tiny{BAKER-ROSETTASERVER\_TS4} &\tiny{Pcons-net\_TS2} &\tiny{MUFOLD-Server\_TS4} \\
\tiny{GDT\_TS = 0.11} &\tiny{GDT\_TS = 0.18} &\tiny{GDT\_TS = 0.24} &\tiny{GDT\_TS = 0.32} \\
\begin{minipage}{\linewidth}\includegraphics[width=\linewidth]{GradCAMOutput/T0812/T0812_RBO_Aleph_TS3.png}\end{minipage}&\begin{minipage}{\linewidth}\includegraphics[width=\linewidth]{GradCAMOutput/T0812/T0812_BAKER-ROSETTASERVER_TS4.png}\end{minipage}&\begin{minipage}{\linewidth}\includegraphics[width=\linewidth]{GradCAMOutput/T0812/T0812_Pcons-net_TS2.png}\end{minipage}&\begin{minipage}{\linewidth}\includegraphics[width=\linewidth]{GradCAMOutput/T0812/T0812_MUFOLD-Server_TS4.png}\end{minipage}\\

\end{tabularx}
%}
\hskip\headheight}
\end{center}	
	\begin{center}
	\makebox[10pt][c]{
	\hskip-\footskip
	\begin{tabularx}{0.95\paperwidth}{X*{4}{p{5.0cm}}}
					\hline
\tiny{T0813} &\tiny{T0813} &\tiny{T0813} &\tiny{T0813} \\
\tiny{Pcons-net\_TS2} &\tiny{FFAS03\_TS1} &\tiny{RBO\_Aleph\_TS3} &\tiny{MULTICOM-CONSTRUCT\_TS1} \\
\tiny{GDT\_TS = 0.66} &\tiny{GDT\_TS = 0.73} &\tiny{GDT\_TS = 0.76} &\tiny{GDT\_TS = 0.79} \\
\begin{minipage}{\linewidth}\includegraphics[width=\linewidth]{GradCAMOutput/T0813/T0813_Pcons-net_TS2.png}\end{minipage}&\begin{minipage}{\linewidth}\includegraphics[width=\linewidth]{GradCAMOutput/T0813/T0813_FFAS03_TS1.png}\end{minipage}&\begin{minipage}{\linewidth}\includegraphics[width=\linewidth]{GradCAMOutput/T0813/T0813_RBO_Aleph_TS3.png}\end{minipage}&\begin{minipage}{\linewidth}\includegraphics[width=\linewidth]{GradCAMOutput/T0813/T0813_MULTICOM-CONSTRUCT_TS1.png}\end{minipage}\\
\hline
\tiny{T0814} &\tiny{T0814} &\tiny{T0814} &\tiny{T0814} \\
\tiny{RBO\_Aleph\_TS3} &\tiny{Distill\_TS3} &\tiny{MULTICOM-CONSTRUCT\_TS1} &\tiny{3D-Jigsaw-V5\_1\_TS4} \\
\tiny{GDT\_TS = 0.05} &\tiny{GDT\_TS = 0.09} &\tiny{GDT\_TS = 0.14} &\tiny{GDT\_TS = 0.18} \\
\begin{minipage}{\linewidth}\includegraphics[width=\linewidth]{GradCAMOutput/T0814/T0814_RBO_Aleph_TS3.png}\end{minipage}&\begin{minipage}{\linewidth}\includegraphics[width=\linewidth]{GradCAMOutput/T0814/T0814_Distill_TS3.png}\end{minipage}&\begin{minipage}{\linewidth}\includegraphics[width=\linewidth]{GradCAMOutput/T0814/T0814_MULTICOM-CONSTRUCT_TS1.png}\end{minipage}&\begin{minipage}{\linewidth}\includegraphics[width=\linewidth]{GradCAMOutput/T0814/T0814_3D-Jigsaw-V5_1_TS4.png}\end{minipage}\\
\hline
\tiny{T0815} &\tiny{T0815} &\tiny{T0815} &\tiny{T0815} \\
\tiny{MUFOLD-Server\_TS4} &\tiny{Seok-server\_TS2} &\tiny{Pcons-net\_TS2} &\tiny{RBO\_Aleph\_TS3} \\
\tiny{GDT\_TS = 0.76} &\tiny{GDT\_TS = 0.80} &\tiny{GDT\_TS = 0.86} &\tiny{GDT\_TS = 0.90} \\
\begin{minipage}{\linewidth}\includegraphics[width=\linewidth]{GradCAMOutput/T0815/T0815_MUFOLD-Server_TS4.png}\end{minipage}&\begin{minipage}{\linewidth}\includegraphics[width=\linewidth]{GradCAMOutput/T0815/T0815_Seok-server_TS2.png}\end{minipage}&\begin{minipage}{\linewidth}\includegraphics[width=\linewidth]{GradCAMOutput/T0815/T0815_Pcons-net_TS2.png}\end{minipage}&\begin{minipage}{\linewidth}\includegraphics[width=\linewidth]{GradCAMOutput/T0815/T0815_RBO_Aleph_TS3.png}\end{minipage}\\
\hline
\tiny{T0816} &\tiny{T0816} &\tiny{T0816} &\tiny{T0816} \\
\tiny{MUFOLD-Server\_TS4} &\tiny{MULTICOM-CONSTRUCT\_TS1} &\tiny{RaptorX-FM\_TS1} &\tiny{RBO\_Aleph\_TS3} \\
\tiny{GDT\_TS = 0.39} &\tiny{GDT\_TS = 0.47} &\tiny{GDT\_TS = 0.51} &\tiny{GDT\_TS = 0.68} \\
\begin{minipage}{\linewidth}\includegraphics[width=\linewidth]{GradCAMOutput/T0816/T0816_MUFOLD-Server_TS4.png}\end{minipage}&\begin{minipage}{\linewidth}\includegraphics[width=\linewidth]{GradCAMOutput/T0816/T0816_MULTICOM-CONSTRUCT_TS1.png}\end{minipage}&\begin{minipage}{\linewidth}\includegraphics[width=\linewidth]{GradCAMOutput/T0816/T0816_RaptorX-FM_TS1.png}\end{minipage}&\begin{minipage}{\linewidth}\includegraphics[width=\linewidth]{GradCAMOutput/T0816/T0816_RBO_Aleph_TS3.png}\end{minipage}\\

\end{tabularx}
%}
\hskip\headheight}
\end{center}	
	\begin{center}
	\makebox[10pt][c]{
	\hskip-\footskip
	\begin{tabularx}{0.95\paperwidth}{X*{4}{p{5.0cm}}}
					\hline
\tiny{T0817} &\tiny{T0817} &\tiny{T0817} &\tiny{T0817} \\
\tiny{MULTICOM-CONSTRUCT\_TS1} &\tiny{MUFOLD-Server\_TS4} &\tiny{BioShell-server\_TS3} &\tiny{FFAS03\_TS1} \\
\tiny{GDT\_TS = 0.43} &\tiny{GDT\_TS = 0.51} &\tiny{GDT\_TS = 0.57} &\tiny{GDT\_TS = 0.62} \\
\begin{minipage}{\linewidth}\includegraphics[width=\linewidth]{GradCAMOutput/T0817/T0817_MULTICOM-CONSTRUCT_TS1.png}\end{minipage}&\begin{minipage}{\linewidth}\includegraphics[width=\linewidth]{GradCAMOutput/T0817/T0817_MUFOLD-Server_TS4.png}\end{minipage}&\begin{minipage}{\linewidth}\includegraphics[width=\linewidth]{GradCAMOutput/T0817/T0817_BioShell-server_TS3.png}\end{minipage}&\begin{minipage}{\linewidth}\includegraphics[width=\linewidth]{GradCAMOutput/T0817/T0817_FFAS03_TS1.png}\end{minipage}\\
\hline
\tiny{T0818} &\tiny{T0818} &\tiny{T0818} &\tiny{T0818} \\
\tiny{RBO\_Aleph\_TS3} &\tiny{MULTICOM-CONSTRUCT\_TS1} &\tiny{MUFOLD-Server\_TS4} &\tiny{Alpha-Gelly-Server\_TS2} \\
\tiny{GDT\_TS = 0.19} &\tiny{GDT\_TS = 0.26} &\tiny{GDT\_TS = 0.26} &\tiny{GDT\_TS = 0.31} \\
\begin{minipage}{\linewidth}\includegraphics[width=\linewidth]{GradCAMOutput/T0818/T0818_RBO_Aleph_TS3.png}\end{minipage}&\begin{minipage}{\linewidth}\includegraphics[width=\linewidth]{GradCAMOutput/T0818/T0818_MULTICOM-CONSTRUCT_TS1.png}\end{minipage}&\begin{minipage}{\linewidth}\includegraphics[width=\linewidth]{GradCAMOutput/T0818/T0818_MUFOLD-Server_TS4.png}\end{minipage}&\begin{minipage}{\linewidth}\includegraphics[width=\linewidth]{GradCAMOutput/T0818/T0818_Alpha-Gelly-Server_TS2.png}\end{minipage}\\
\hline
\tiny{T0819} &\tiny{T0819} &\tiny{T0819} &\tiny{T0819} \\
\tiny{slbio\_TS3} &\tiny{MULTICOM-CONSTRUCT\_TS1} &\tiny{FFAS03\_TS1} &\tiny{BAKER-ROSETTASERVER\_TS4} \\
\tiny{GDT\_TS = 0.68} &\tiny{GDT\_TS = 0.73} &\tiny{GDT\_TS = 0.74} &\tiny{GDT\_TS = 0.83} \\
\begin{minipage}{\linewidth}\includegraphics[width=\linewidth]{GradCAMOutput/T0819/T0819_slbio_TS3.png}\end{minipage}&\begin{minipage}{\linewidth}\includegraphics[width=\linewidth]{GradCAMOutput/T0819/T0819_MULTICOM-CONSTRUCT_TS1.png}\end{minipage}&\begin{minipage}{\linewidth}\includegraphics[width=\linewidth]{GradCAMOutput/T0819/T0819_FFAS03_TS1.png}\end{minipage}&\begin{minipage}{\linewidth}\includegraphics[width=\linewidth]{GradCAMOutput/T0819/T0819_BAKER-ROSETTASERVER_TS4.png}\end{minipage}\\
\hline
\tiny{T0820} &\tiny{T0820} &\tiny{T0820} &\tiny{T0820} \\
\tiny{MULTICOM-CONSTRUCT\_TS1} &\tiny{RBO\_Aleph\_TS3} &\tiny{Zhang-Server\_TS1} &\tiny{QUARK\_TS4} \\
\tiny{GDT\_TS = 0.19} &\tiny{GDT\_TS = 0.23} &\tiny{GDT\_TS = 0.28} &\tiny{GDT\_TS = 0.31} \\
\begin{minipage}{\linewidth}\includegraphics[width=\linewidth]{GradCAMOutput/T0820/T0820_MULTICOM-CONSTRUCT_TS1.png}\end{minipage}&\begin{minipage}{\linewidth}\includegraphics[width=\linewidth]{GradCAMOutput/T0820/T0820_RBO_Aleph_TS3.png}\end{minipage}&\begin{minipage}{\linewidth}\includegraphics[width=\linewidth]{GradCAMOutput/T0820/T0820_Zhang-Server_TS1.png}\end{minipage}&\begin{minipage}{\linewidth}\includegraphics[width=\linewidth]{GradCAMOutput/T0820/T0820_QUARK_TS4.png}\end{minipage}\\

\end{tabularx}
%}
\hskip\headheight}
\end{center}	
	\begin{center}
	\makebox[10pt][c]{
	\hskip-\footskip
	\begin{tabularx}{0.95\paperwidth}{X*{4}{p{5.0cm}}}
					\hline
\tiny{T0821} &\tiny{T0821} &\tiny{T0821} &\tiny{T0821} \\
\tiny{FFAS03\_TS1} &\tiny{nns\_TS5} &\tiny{RBO\_Aleph\_TS3} &\tiny{BAKER-ROSETTASERVER\_TS4} \\
\tiny{GDT\_TS = 0.35} &\tiny{GDT\_TS = 0.40} &\tiny{GDT\_TS = 0.51} &\tiny{GDT\_TS = 0.60} \\
\begin{minipage}{\linewidth}\includegraphics[width=\linewidth]{GradCAMOutput/T0821/T0821_FFAS03_TS1.png}\end{minipage}&\begin{minipage}{\linewidth}\includegraphics[width=\linewidth]{GradCAMOutput/T0821/T0821_nns_TS5.png}\end{minipage}&\begin{minipage}{\linewidth}\includegraphics[width=\linewidth]{GradCAMOutput/T0821/T0821_RBO_Aleph_TS3.png}\end{minipage}&\begin{minipage}{\linewidth}\includegraphics[width=\linewidth]{GradCAMOutput/T0821/T0821_BAKER-ROSETTASERVER_TS4.png}\end{minipage}\\
\hline
\tiny{T0822} &\tiny{T0822} &\tiny{T0822} &\tiny{T0822} \\
\tiny{RBO\_Aleph\_TS3} &\tiny{BioShell-server\_TS3} &\tiny{FALCON\_MANUAL\_TS2} &\tiny{MULTICOM-CONSTRUCT\_TS1} \\
\tiny{GDT\_TS = 0.23} &\tiny{GDT\_TS = 0.28} &\tiny{GDT\_TS = 0.40} &\tiny{GDT\_TS = 0.45} \\
\begin{minipage}{\linewidth}\includegraphics[width=\linewidth]{GradCAMOutput/T0822/T0822_RBO_Aleph_TS3.png}\end{minipage}&\begin{minipage}{\linewidth}\includegraphics[width=\linewidth]{GradCAMOutput/T0822/T0822_BioShell-server_TS3.png}\end{minipage}&\begin{minipage}{\linewidth}\includegraphics[width=\linewidth]{GradCAMOutput/T0822/T0822_FALCON_MANUAL_TS2.png}\end{minipage}&\begin{minipage}{\linewidth}\includegraphics[width=\linewidth]{GradCAMOutput/T0822/T0822_MULTICOM-CONSTRUCT_TS1.png}\end{minipage}\\
\hline
\tiny{T0823} &\tiny{T0823} &\tiny{T0823} &\tiny{T0823} \\
\tiny{ZHOU-SPARKS-X\_TS5} &\tiny{BioShell-server\_TS3} &\tiny{FFAS03\_TS1} &\tiny{BAKER-ROSETTASERVER\_TS4} \\
\tiny{GDT\_TS = 0.52} &\tiny{GDT\_TS = 0.54} &\tiny{GDT\_TS = 0.58} &\tiny{GDT\_TS = 0.60} \\
\begin{minipage}{\linewidth}\includegraphics[width=\linewidth]{GradCAMOutput/T0823/T0823_ZHOU-SPARKS-X_TS5.png}\end{minipage}&\begin{minipage}{\linewidth}\includegraphics[width=\linewidth]{GradCAMOutput/T0823/T0823_BioShell-server_TS3.png}\end{minipage}&\begin{minipage}{\linewidth}\includegraphics[width=\linewidth]{GradCAMOutput/T0823/T0823_FFAS03_TS1.png}\end{minipage}&\begin{minipage}{\linewidth}\includegraphics[width=\linewidth]{GradCAMOutput/T0823/T0823_BAKER-ROSETTASERVER_TS4.png}\end{minipage}\\
\hline
\tiny{T0824} &\tiny{T0824} &\tiny{T0824} &\tiny{T0824} \\
\tiny{BioShell-server\_TS3} &\tiny{Distill\_TS3} &\tiny{RBO\_Aleph\_TS3} &\tiny{Zhang-Server\_TS1} \\
\tiny{GDT\_TS = 0.20} &\tiny{GDT\_TS = 0.23} &\tiny{GDT\_TS = 0.26} &\tiny{GDT\_TS = 0.29} \\
\begin{minipage}{\linewidth}\includegraphics[width=\linewidth]{GradCAMOutput/T0824/T0824_BioShell-server_TS3.png}\end{minipage}&\begin{minipage}{\linewidth}\includegraphics[width=\linewidth]{GradCAMOutput/T0824/T0824_Distill_TS3.png}\end{minipage}&\begin{minipage}{\linewidth}\includegraphics[width=\linewidth]{GradCAMOutput/T0824/T0824_RBO_Aleph_TS3.png}\end{minipage}&\begin{minipage}{\linewidth}\includegraphics[width=\linewidth]{GradCAMOutput/T0824/T0824_Zhang-Server_TS1.png}\end{minipage}\\

\end{tabularx}
%}
\hskip\headheight}
\end{center}	
	\begin{center}
	\makebox[10pt][c]{
	\hskip-\footskip
	\begin{tabularx}{0.95\paperwidth}{X*{4}{p{5.0cm}}}
					\hline
\tiny{T0825} &\tiny{T0825} &\tiny{T0825} &\tiny{T0825} \\
\tiny{myprotein-me\_TS4} &\tiny{BioShell-server\_TS3} &\tiny{FFAS03\_TS1} &\tiny{MULTICOM-CONSTRUCT\_TS1} \\
\tiny{GDT\_TS = 0.61} &\tiny{GDT\_TS = 0.67} &\tiny{GDT\_TS = 0.83} &\tiny{GDT\_TS = 0.84} \\
\begin{minipage}{\linewidth}\includegraphics[width=\linewidth]{GradCAMOutput/T0825/T0825_myprotein-me_TS4.png}\end{minipage}&\begin{minipage}{\linewidth}\includegraphics[width=\linewidth]{GradCAMOutput/T0825/T0825_BioShell-server_TS3.png}\end{minipage}&\begin{minipage}{\linewidth}\includegraphics[width=\linewidth]{GradCAMOutput/T0825/T0825_FFAS03_TS1.png}\end{minipage}&\begin{minipage}{\linewidth}\includegraphics[width=\linewidth]{GradCAMOutput/T0825/T0825_MULTICOM-CONSTRUCT_TS1.png}\end{minipage}\\
\hline
\tiny{T0827} &\tiny{T0827} &\tiny{T0827} &\tiny{T0827} \\
\tiny{RBO\_Aleph\_TS3} &\tiny{BAKER-ROSETTASERVER\_TS4} &\tiny{Zhang-Server\_TS1} &\tiny{nns\_TS1} \\
\tiny{GDT\_TS = 0.10} &\tiny{GDT\_TS = 0.14} &\tiny{GDT\_TS = 0.21} &\tiny{GDT\_TS = 0.24} \\
\begin{minipage}{\linewidth}\includegraphics[width=\linewidth]{GradCAMOutput/T0827/T0827_RBO_Aleph_TS3.png}\end{minipage}&\begin{minipage}{\linewidth}\includegraphics[width=\linewidth]{GradCAMOutput/T0827/T0827_BAKER-ROSETTASERVER_TS4.png}\end{minipage}&\begin{minipage}{\linewidth}\includegraphics[width=\linewidth]{GradCAMOutput/T0827/T0827_Zhang-Server_TS1.png}\end{minipage}&\begin{minipage}{\linewidth}\includegraphics[width=\linewidth]{GradCAMOutput/T0827/T0827_nns_TS1.png}\end{minipage}\\
\hline
\tiny{T0829} &\tiny{T0829} &\tiny{T0829} &\tiny{T0829} \\
\tiny{Alpha-Gelly-Server\_TS2} &\tiny{MULTICOM-CONSTRUCT\_TS1} &\tiny{RBO\_Aleph\_TS3} &\tiny{Zhang-Server\_TS1} \\
\tiny{GDT\_TS = 0.35} &\tiny{GDT\_TS = 0.46} &\tiny{GDT\_TS = 0.56} &\tiny{GDT\_TS = 0.62} \\
\begin{minipage}{\linewidth}\includegraphics[width=\linewidth]{GradCAMOutput/T0829/T0829_Alpha-Gelly-Server_TS2.png}\end{minipage}&\begin{minipage}{\linewidth}\includegraphics[width=\linewidth]{GradCAMOutput/T0829/T0829_MULTICOM-CONSTRUCT_TS1.png}\end{minipage}&\begin{minipage}{\linewidth}\includegraphics[width=\linewidth]{GradCAMOutput/T0829/T0829_RBO_Aleph_TS3.png}\end{minipage}&\begin{minipage}{\linewidth}\includegraphics[width=\linewidth]{GradCAMOutput/T0829/T0829_Zhang-Server_TS1.png}\end{minipage}\\
\hline
\tiny{T0830} &\tiny{T0830} &\tiny{T0830} &\tiny{T0830} \\
\tiny{RBO\_Aleph\_TS3} &\tiny{RBO\_Aleph\_TS4} &\tiny{MULTICOM-CONSTRUCT\_TS1} &\tiny{FFAS03\_TS1} \\
\tiny{GDT\_TS = 0.09} &\tiny{GDT\_TS = 0.17} &\tiny{GDT\_TS = 0.27} &\tiny{GDT\_TS = 0.33} \\
\begin{minipage}{\linewidth}\includegraphics[width=\linewidth]{GradCAMOutput/T0830/T0830_RBO_Aleph_TS3.png}\end{minipage}&\begin{minipage}{\linewidth}\includegraphics[width=\linewidth]{GradCAMOutput/T0830/T0830_RBO_Aleph_TS4.png}\end{minipage}&\begin{minipage}{\linewidth}\includegraphics[width=\linewidth]{GradCAMOutput/T0830/T0830_MULTICOM-CONSTRUCT_TS1.png}\end{minipage}&\begin{minipage}{\linewidth}\includegraphics[width=\linewidth]{GradCAMOutput/T0830/T0830_FFAS03_TS1.png}\end{minipage}\\

\end{tabularx}
%}
\hskip\headheight}
\end{center}	
	\begin{center}
	\makebox[10pt][c]{
	\hskip-\footskip
	\begin{tabularx}{0.95\paperwidth}{X*{4}{p{5.0cm}}}
					\hline
\tiny{T0831} &\tiny{T0831} &\tiny{T0831} &\tiny{T0831} \\
\tiny{BioShell-server\_TS3} &\tiny{MUFOLD-Server\_TS4} &\tiny{RBO\_Aleph\_TS3} &\tiny{Pcons-net\_TS1} \\
\tiny{GDT\_TS = 0.09} &\tiny{GDT\_TS = 0.12} &\tiny{GDT\_TS = 0.13} &\tiny{GDT\_TS = 0.14} \\
\begin{minipage}{\linewidth}\includegraphics[width=\linewidth]{GradCAMOutput/T0831/T0831_BioShell-server_TS3.png}\end{minipage}&\begin{minipage}{\linewidth}\includegraphics[width=\linewidth]{GradCAMOutput/T0831/T0831_MUFOLD-Server_TS4.png}\end{minipage}&\begin{minipage}{\linewidth}\includegraphics[width=\linewidth]{GradCAMOutput/T0831/T0831_RBO_Aleph_TS3.png}\end{minipage}&\begin{minipage}{\linewidth}\includegraphics[width=\linewidth]{GradCAMOutput/T0831/T0831_Pcons-net_TS1.png}\end{minipage}\\
\hline
\tiny{T0832} &\tiny{T0832} &\tiny{T0832} &\tiny{T0832} \\
\tiny{BioShell-server\_TS3} &\tiny{RBO\_Aleph\_TS3} &\tiny{MULTICOM-CONSTRUCT\_TS1} &\tiny{BAKER-ROSETTASERVER\_TS2} \\
\tiny{GDT\_TS = 0.10} &\tiny{GDT\_TS = 0.14} &\tiny{GDT\_TS = 0.15} &\tiny{GDT\_TS = 0.18} \\
\begin{minipage}{\linewidth}\includegraphics[width=\linewidth]{GradCAMOutput/T0832/T0832_BioShell-server_TS3.png}\end{minipage}&\begin{minipage}{\linewidth}\includegraphics[width=\linewidth]{GradCAMOutput/T0832/T0832_RBO_Aleph_TS3.png}\end{minipage}&\begin{minipage}{\linewidth}\includegraphics[width=\linewidth]{GradCAMOutput/T0832/T0832_MULTICOM-CONSTRUCT_TS1.png}\end{minipage}&\begin{minipage}{\linewidth}\includegraphics[width=\linewidth]{GradCAMOutput/T0832/T0832_BAKER-ROSETTASERVER_TS2.png}\end{minipage}\\
\hline
\tiny{T0833} &\tiny{T0833} &\tiny{T0833} &\tiny{T0833} \\
\tiny{BioSerf\_TS3} &\tiny{RBO\_Aleph\_TS3} &\tiny{MULTICOM-CONSTRUCT\_TS1} &\tiny{FFAS03\_TS1} \\
\tiny{GDT\_TS = 0.28} &\tiny{GDT\_TS = 0.45} &\tiny{GDT\_TS = 0.50} &\tiny{GDT\_TS = 0.62} \\
\begin{minipage}{\linewidth}\includegraphics[width=\linewidth]{GradCAMOutput/T0833/T0833_BioSerf_TS3.png}\end{minipage}&\begin{minipage}{\linewidth}\includegraphics[width=\linewidth]{GradCAMOutput/T0833/T0833_RBO_Aleph_TS3.png}\end{minipage}&\begin{minipage}{\linewidth}\includegraphics[width=\linewidth]{GradCAMOutput/T0833/T0833_MULTICOM-CONSTRUCT_TS1.png}\end{minipage}&\begin{minipage}{\linewidth}\includegraphics[width=\linewidth]{GradCAMOutput/T0833/T0833_FFAS03_TS1.png}\end{minipage}\\
\hline
\tiny{T0834} &\tiny{T0834} &\tiny{T0834} &\tiny{T0834} \\
\tiny{MULTICOM-CONSTRUCT\_TS1} &\tiny{BAKER-ROSETTASERVER\_TS4} &\tiny{RBO\_Aleph\_TS4} &\tiny{RBO\_Aleph\_TS3} \\
\tiny{GDT\_TS = 0.10} &\tiny{GDT\_TS = 0.11} &\tiny{GDT\_TS = 0.16} &\tiny{GDT\_TS = 0.19} \\
\begin{minipage}{\linewidth}\includegraphics[width=\linewidth]{GradCAMOutput/T0834/T0834_MULTICOM-CONSTRUCT_TS1.png}\end{minipage}&\begin{minipage}{\linewidth}\includegraphics[width=\linewidth]{GradCAMOutput/T0834/T0834_BAKER-ROSETTASERVER_TS4.png}\end{minipage}&\begin{minipage}{\linewidth}\includegraphics[width=\linewidth]{GradCAMOutput/T0834/T0834_RBO_Aleph_TS4.png}\end{minipage}&\begin{minipage}{\linewidth}\includegraphics[width=\linewidth]{GradCAMOutput/T0834/T0834_RBO_Aleph_TS3.png}\end{minipage}\\

\end{tabularx}
%}
\hskip\headheight}
\end{center}	
	\begin{center}
	\makebox[10pt][c]{
	\hskip-\footskip
	\begin{tabularx}{0.95\paperwidth}{X*{4}{p{5.0cm}}}
					\hline
\tiny{T0835} &\tiny{T0835} &\tiny{T0835} &\tiny{T0835} \\
\tiny{BioShell-server\_TS3} &\tiny{MUFOLD-Server\_TS4} &\tiny{FFAS03\_TS1} &\tiny{BAKER-ROSETTASERVER\_TS4} \\
\tiny{GDT\_TS = 0.28} &\tiny{GDT\_TS = 0.33} &\tiny{GDT\_TS = 0.41} &\tiny{GDT\_TS = 0.45} \\
\begin{minipage}{\linewidth}\includegraphics[width=\linewidth]{GradCAMOutput/T0835/T0835_BioShell-server_TS3.png}\end{minipage}&\begin{minipage}{\linewidth}\includegraphics[width=\linewidth]{GradCAMOutput/T0835/T0835_MUFOLD-Server_TS4.png}\end{minipage}&\begin{minipage}{\linewidth}\includegraphics[width=\linewidth]{GradCAMOutput/T0835/T0835_FFAS03_TS1.png}\end{minipage}&\begin{minipage}{\linewidth}\includegraphics[width=\linewidth]{GradCAMOutput/T0835/T0835_BAKER-ROSETTASERVER_TS4.png}\end{minipage}\\
\hline
\tiny{T0836} &\tiny{T0836} &\tiny{T0836} &\tiny{T0836} \\
\tiny{BioShell-server\_TS3} &\tiny{FFAS03\_TS1} &\tiny{BAKER-ROSETTASERVER\_TS4} &\tiny{Zhang-Server\_TS3} \\
\tiny{GDT\_TS = 0.17} &\tiny{GDT\_TS = 0.20} &\tiny{GDT\_TS = 0.22} &\tiny{GDT\_TS = 0.30} \\
\begin{minipage}{\linewidth}\includegraphics[width=\linewidth]{GradCAMOutput/T0836/T0836_BioShell-server_TS3.png}\end{minipage}&\begin{minipage}{\linewidth}\includegraphics[width=\linewidth]{GradCAMOutput/T0836/T0836_FFAS03_TS1.png}\end{minipage}&\begin{minipage}{\linewidth}\includegraphics[width=\linewidth]{GradCAMOutput/T0836/T0836_BAKER-ROSETTASERVER_TS4.png}\end{minipage}&\begin{minipage}{\linewidth}\includegraphics[width=\linewidth]{GradCAMOutput/T0836/T0836_Zhang-Server_TS3.png}\end{minipage}\\
\hline
\tiny{T0837} &\tiny{T0837} &\tiny{T0837} &\tiny{T0837} \\
\tiny{BioShell-server\_TS3} &\tiny{RBO\_Aleph\_TS3} &\tiny{MULTICOM-CLUSTER\_TS2} &\tiny{Zhang-Server\_TS2} \\
\tiny{GDT\_TS = 0.24} &\tiny{GDT\_TS = 0.36} &\tiny{GDT\_TS = 0.47} &\tiny{GDT\_TS = 0.58} \\
\begin{minipage}{\linewidth}\includegraphics[width=\linewidth]{GradCAMOutput/T0837/T0837_BioShell-server_TS3.png}\end{minipage}&\begin{minipage}{\linewidth}\includegraphics[width=\linewidth]{GradCAMOutput/T0837/T0837_RBO_Aleph_TS3.png}\end{minipage}&\begin{minipage}{\linewidth}\includegraphics[width=\linewidth]{GradCAMOutput/T0837/T0837_MULTICOM-CLUSTER_TS2.png}\end{minipage}&\begin{minipage}{\linewidth}\includegraphics[width=\linewidth]{GradCAMOutput/T0837/T0837_Zhang-Server_TS2.png}\end{minipage}\\
\hline
\tiny{T0838} &\tiny{T0838} &\tiny{T0838} &\tiny{T0838} \\
\tiny{RBO\_Aleph\_TS3} &\tiny{BioSerf\_TS4} &\tiny{MULTICOM-CONSTRUCT\_TS1} &\tiny{FFAS03\_TS1} \\
\tiny{GDT\_TS = 0.19} &\tiny{GDT\_TS = 0.22} &\tiny{GDT\_TS = 0.36} &\tiny{GDT\_TS = 0.43} \\
\begin{minipage}{\linewidth}\includegraphics[width=\linewidth]{GradCAMOutput/T0838/T0838_RBO_Aleph_TS3.png}\end{minipage}&\begin{minipage}{\linewidth}\includegraphics[width=\linewidth]{GradCAMOutput/T0838/T0838_BioSerf_TS4.png}\end{minipage}&\begin{minipage}{\linewidth}\includegraphics[width=\linewidth]{GradCAMOutput/T0838/T0838_MULTICOM-CONSTRUCT_TS1.png}\end{minipage}&\begin{minipage}{\linewidth}\includegraphics[width=\linewidth]{GradCAMOutput/T0838/T0838_FFAS03_TS1.png}\end{minipage}\\

\end{tabularx}
%}
\hskip\headheight}
\end{center}	
	\begin{center}
	\makebox[10pt][c]{
	\hskip-\footskip
	\begin{tabularx}{0.95\paperwidth}{X*{4}{p{5.0cm}}}
					\hline
\tiny{T0840} &\tiny{T0840} &\tiny{T0840} &\tiny{T0840} \\
\tiny{BAKER-ROSETTASERVER\_TS4} &\tiny{BioSerf\_TS4} &\tiny{RBO\_Aleph\_TS3} &\tiny{FFAS03\_TS1} \\
\tiny{GDT\_TS = 0.53} &\tiny{GDT\_TS = 0.67} &\tiny{GDT\_TS = 0.69} &\tiny{GDT\_TS = 0.90} \\
\begin{minipage}{\linewidth}\includegraphics[width=\linewidth]{GradCAMOutput/T0840/T0840_BAKER-ROSETTASERVER_TS4.png}\end{minipage}&\begin{minipage}{\linewidth}\includegraphics[width=\linewidth]{GradCAMOutput/T0840/T0840_BioSerf_TS4.png}\end{minipage}&\begin{minipage}{\linewidth}\includegraphics[width=\linewidth]{GradCAMOutput/T0840/T0840_RBO_Aleph_TS3.png}\end{minipage}&\begin{minipage}{\linewidth}\includegraphics[width=\linewidth]{GradCAMOutput/T0840/T0840_FFAS03_TS1.png}\end{minipage}\\
\hline
\tiny{T0841} &\tiny{T0841} &\tiny{T0841} &\tiny{T0841} \\
\tiny{BioShell-server\_TS3} &\tiny{MUFOLD-Server\_TS4} &\tiny{RBO\_Aleph\_TS3} &\tiny{FFAS03\_TS1} \\
\tiny{GDT\_TS = 0.71} &\tiny{GDT\_TS = 0.75} &\tiny{GDT\_TS = 0.79} &\tiny{GDT\_TS = 0.90} \\
\begin{minipage}{\linewidth}\includegraphics[width=\linewidth]{GradCAMOutput/T0841/T0841_BioShell-server_TS3.png}\end{minipage}&\begin{minipage}{\linewidth}\includegraphics[width=\linewidth]{GradCAMOutput/T0841/T0841_MUFOLD-Server_TS4.png}\end{minipage}&\begin{minipage}{\linewidth}\includegraphics[width=\linewidth]{GradCAMOutput/T0841/T0841_RBO_Aleph_TS3.png}\end{minipage}&\begin{minipage}{\linewidth}\includegraphics[width=\linewidth]{GradCAMOutput/T0841/T0841_FFAS03_TS1.png}\end{minipage}\\
\hline
\tiny{T0843} &\tiny{T0843} &\tiny{T0843} &\tiny{T0843} \\
\tiny{FFAS03\_TS1} &\tiny{MUFOLD-Server\_TS4} &\tiny{RBO\_Aleph\_TS3} &\tiny{MULTICOM-CONSTRUCT\_TS1} \\
\tiny{GDT\_TS = 0.71} &\tiny{GDT\_TS = 0.72} &\tiny{GDT\_TS = 0.76} &\tiny{GDT\_TS = 0.78} \\
\begin{minipage}{\linewidth}\includegraphics[width=\linewidth]{GradCAMOutput/T0843/T0843_FFAS03_TS1.png}\end{minipage}&\begin{minipage}{\linewidth}\includegraphics[width=\linewidth]{GradCAMOutput/T0843/T0843_MUFOLD-Server_TS4.png}\end{minipage}&\begin{minipage}{\linewidth}\includegraphics[width=\linewidth]{GradCAMOutput/T0843/T0843_RBO_Aleph_TS3.png}\end{minipage}&\begin{minipage}{\linewidth}\includegraphics[width=\linewidth]{GradCAMOutput/T0843/T0843_MULTICOM-CONSTRUCT_TS1.png}\end{minipage}\\
\hline
\tiny{T0845} &\tiny{T0845} &\tiny{T0845} &\tiny{T0845} \\
\tiny{MUFOLD-Server\_TS4} &\tiny{Atome2\_CBS\_TS3} &\tiny{RBO\_Aleph\_TS3} &\tiny{MULTICOM-CONSTRUCT\_TS1} \\
\tiny{GDT\_TS = 0.25} &\tiny{GDT\_TS = 0.30} &\tiny{GDT\_TS = 0.36} &\tiny{GDT\_TS = 0.40} \\
\begin{minipage}{\linewidth}\includegraphics[width=\linewidth]{GradCAMOutput/T0845/T0845_MUFOLD-Server_TS4.png}\end{minipage}&\begin{minipage}{\linewidth}\includegraphics[width=\linewidth]{GradCAMOutput/T0845/T0845_Atome2_CBS_TS3.png}\end{minipage}&\begin{minipage}{\linewidth}\includegraphics[width=\linewidth]{GradCAMOutput/T0845/T0845_RBO_Aleph_TS3.png}\end{minipage}&\begin{minipage}{\linewidth}\includegraphics[width=\linewidth]{GradCAMOutput/T0845/T0845_MULTICOM-CONSTRUCT_TS1.png}\end{minipage}\\

\end{tabularx}
%}
\hskip\headheight}
\end{center}	
	\begin{center}
	\makebox[10pt][c]{
	\hskip-\footskip
	\begin{tabularx}{0.95\paperwidth}{X*{4}{p{5.0cm}}}
					\hline
\tiny{T0847} &\tiny{T0847} &\tiny{T0847} &\tiny{T0847} \\
\tiny{Pcons-net\_TS2} &\tiny{Seok-server\_TS2} &\tiny{RBO\_Aleph\_TS3} &\tiny{Atome2\_CBS\_TS3} \\
\tiny{GDT\_TS = 0.63} &\tiny{GDT\_TS = 0.66} &\tiny{GDT\_TS = 0.70} &\tiny{GDT\_TS = 0.72} \\
\begin{minipage}{\linewidth}\includegraphics[width=\linewidth]{GradCAMOutput/T0847/T0847_Pcons-net_TS2.png}\end{minipage}&\begin{minipage}{\linewidth}\includegraphics[width=\linewidth]{GradCAMOutput/T0847/T0847_Seok-server_TS2.png}\end{minipage}&\begin{minipage}{\linewidth}\includegraphics[width=\linewidth]{GradCAMOutput/T0847/T0847_RBO_Aleph_TS3.png}\end{minipage}&\begin{minipage}{\linewidth}\includegraphics[width=\linewidth]{GradCAMOutput/T0847/T0847_Atome2_CBS_TS3.png}\end{minipage}\\
\hline
\tiny{T0848} &\tiny{T0848} &\tiny{T0848} &\tiny{T0848} \\
\tiny{Alpha-Gelly-Server\_TS2} &\tiny{Distill\_TS3} &\tiny{RBO\_Aleph\_TS3} &\tiny{BAKER-ROSETTASERVER\_TS4} \\
\tiny{GDT\_TS = 0.08} &\tiny{GDT\_TS = 0.18} &\tiny{GDT\_TS = 0.21} &\tiny{GDT\_TS = 0.29} \\
\begin{minipage}{\linewidth}\includegraphics[width=\linewidth]{GradCAMOutput/T0848/T0848_Alpha-Gelly-Server_TS2.png}\end{minipage}&\begin{minipage}{\linewidth}\includegraphics[width=\linewidth]{GradCAMOutput/T0848/T0848_Distill_TS3.png}\end{minipage}&\begin{minipage}{\linewidth}\includegraphics[width=\linewidth]{GradCAMOutput/T0848/T0848_RBO_Aleph_TS3.png}\end{minipage}&\begin{minipage}{\linewidth}\includegraphics[width=\linewidth]{GradCAMOutput/T0848/T0848_BAKER-ROSETTASERVER_TS4.png}\end{minipage}\\
\hline
\tiny{T0849} &\tiny{T0849} &\tiny{T0849} &\tiny{T0849} \\
\tiny{BioShell-server\_TS3} &\tiny{MUFOLD-Server\_TS4} &\tiny{MULTICOM-CONSTRUCT\_TS1} &\tiny{FFAS03\_TS1} \\
\tiny{GDT\_TS = 0.48} &\tiny{GDT\_TS = 0.54} &\tiny{GDT\_TS = 0.62} &\tiny{GDT\_TS = 0.63} \\
\begin{minipage}{\linewidth}\includegraphics[width=\linewidth]{GradCAMOutput/T0849/T0849_BioShell-server_TS3.png}\end{minipage}&\begin{minipage}{\linewidth}\includegraphics[width=\linewidth]{GradCAMOutput/T0849/T0849_MUFOLD-Server_TS4.png}\end{minipage}&\begin{minipage}{\linewidth}\includegraphics[width=\linewidth]{GradCAMOutput/T0849/T0849_MULTICOM-CONSTRUCT_TS1.png}\end{minipage}&\begin{minipage}{\linewidth}\includegraphics[width=\linewidth]{GradCAMOutput/T0849/T0849_FFAS03_TS1.png}\end{minipage}\\
\hline
\tiny{T0851} &\tiny{T0851} &\tiny{T0851} &\tiny{T0851} \\
\tiny{BioShell-server\_TS3} &\tiny{Atome2\_CBS\_TS3} &\tiny{MULTICOM-CONSTRUCT\_TS1} &\tiny{RBO\_Aleph\_TS3} \\
\tiny{GDT\_TS = 0.55} &\tiny{GDT\_TS = 0.60} &\tiny{GDT\_TS = 0.69} &\tiny{GDT\_TS = 0.79} \\
\begin{minipage}{\linewidth}\includegraphics[width=\linewidth]{GradCAMOutput/T0851/T0851_BioShell-server_TS3.png}\end{minipage}&\begin{minipage}{\linewidth}\includegraphics[width=\linewidth]{GradCAMOutput/T0851/T0851_Atome2_CBS_TS3.png}\end{minipage}&\begin{minipage}{\linewidth}\includegraphics[width=\linewidth]{GradCAMOutput/T0851/T0851_MULTICOM-CONSTRUCT_TS1.png}\end{minipage}&\begin{minipage}{\linewidth}\includegraphics[width=\linewidth]{GradCAMOutput/T0851/T0851_RBO_Aleph_TS3.png}\end{minipage}\\

\end{tabularx}
%}
\hskip\headheight}
\end{center}	
	\begin{center}
	\makebox[10pt][c]{
	\hskip-\footskip
	\begin{tabularx}{0.95\paperwidth}{X*{4}{p{5.0cm}}}
					\hline
\tiny{T0852} &\tiny{T0852} &\tiny{T0852} &\tiny{T0852} \\
\tiny{RBO\_Aleph\_TS3} &\tiny{Alpha-Gelly-Server\_TS2} &\tiny{Distill\_TS3} &\tiny{MULTICOM-CONSTRUCT\_TS1} \\
\tiny{GDT\_TS = 0.30} &\tiny{GDT\_TS = 0.39} &\tiny{GDT\_TS = 0.40} &\tiny{GDT\_TS = 0.46} \\
\begin{minipage}{\linewidth}\includegraphics[width=\linewidth]{GradCAMOutput/T0852/T0852_RBO_Aleph_TS3.png}\end{minipage}&\begin{minipage}{\linewidth}\includegraphics[width=\linewidth]{GradCAMOutput/T0852/T0852_Alpha-Gelly-Server_TS2.png}\end{minipage}&\begin{minipage}{\linewidth}\includegraphics[width=\linewidth]{GradCAMOutput/T0852/T0852_Distill_TS3.png}\end{minipage}&\begin{minipage}{\linewidth}\includegraphics[width=\linewidth]{GradCAMOutput/T0852/T0852_MULTICOM-CONSTRUCT_TS1.png}\end{minipage}\\
\hline
\tiny{T0853} &\tiny{T0853} &\tiny{T0853} &\tiny{T0853} \\
\tiny{Alpha-Gelly-Server\_TS2} &\tiny{FALCON\_MANUAL\_TS2} &\tiny{RBO\_Aleph\_TS3} &\tiny{Zhang-Server\_TS1} \\
\tiny{GDT\_TS = 0.15} &\tiny{GDT\_TS = 0.23} &\tiny{GDT\_TS = 0.30} &\tiny{GDT\_TS = 0.37} \\
\begin{minipage}{\linewidth}\includegraphics[width=\linewidth]{GradCAMOutput/T0853/T0853_Alpha-Gelly-Server_TS2.png}\end{minipage}&\begin{minipage}{\linewidth}\includegraphics[width=\linewidth]{GradCAMOutput/T0853/T0853_FALCON_MANUAL_TS2.png}\end{minipage}&\begin{minipage}{\linewidth}\includegraphics[width=\linewidth]{GradCAMOutput/T0853/T0853_RBO_Aleph_TS3.png}\end{minipage}&\begin{minipage}{\linewidth}\includegraphics[width=\linewidth]{GradCAMOutput/T0853/T0853_Zhang-Server_TS1.png}\end{minipage}\\
\hline
\tiny{T0854} &\tiny{T0854} &\tiny{T0854} &\tiny{T0854} \\
\tiny{BioShell-server\_TS3} &\tiny{MULTICOM-CONSTRUCT\_TS1} &\tiny{FFAS03\_TS1} &\tiny{RBO\_Aleph\_TS3} \\
\tiny{GDT\_TS = 0.65} &\tiny{GDT\_TS = 0.68} &\tiny{GDT\_TS = 0.69} &\tiny{GDT\_TS = 0.74} \\
\begin{minipage}{\linewidth}\includegraphics[width=\linewidth]{GradCAMOutput/T0854/T0854_BioShell-server_TS3.png}\end{minipage}&\begin{minipage}{\linewidth}\includegraphics[width=\linewidth]{GradCAMOutput/T0854/T0854_MULTICOM-CONSTRUCT_TS1.png}\end{minipage}&\begin{minipage}{\linewidth}\includegraphics[width=\linewidth]{GradCAMOutput/T0854/T0854_FFAS03_TS1.png}\end{minipage}&\begin{minipage}{\linewidth}\includegraphics[width=\linewidth]{GradCAMOutput/T0854/T0854_RBO_Aleph_TS3.png}\end{minipage}\\
\hline
\tiny{T0855} &\tiny{T0855} &\tiny{T0855} &\tiny{T0855} \\
\tiny{BioShell-server\_TS3} &\tiny{Pcons-net\_TS2} &\tiny{RBO\_Aleph\_TS3} &\tiny{Zhang-Server\_TS1} \\
\tiny{GDT\_TS = 0.20} &\tiny{GDT\_TS = 0.28} &\tiny{GDT\_TS = 0.37} &\tiny{GDT\_TS = 0.44} \\
\begin{minipage}{\linewidth}\includegraphics[width=\linewidth]{GradCAMOutput/T0855/T0855_BioShell-server_TS3.png}\end{minipage}&\begin{minipage}{\linewidth}\includegraphics[width=\linewidth]{GradCAMOutput/T0855/T0855_Pcons-net_TS2.png}\end{minipage}&\begin{minipage}{\linewidth}\includegraphics[width=\linewidth]{GradCAMOutput/T0855/T0855_RBO_Aleph_TS3.png}\end{minipage}&\begin{minipage}{\linewidth}\includegraphics[width=\linewidth]{GradCAMOutput/T0855/T0855_Zhang-Server_TS1.png}\end{minipage}\\

\end{tabularx}
%}
\hskip\headheight}
\end{center}
%\end{document}


\begin{figure}[h]
    \centerline{\includegraphics[width=0.7\linewidth]{Fig/image8}}
    \caption{3DCNN score versus GDT\_TS for the two most correlated
      targets and the two least correlated targets in the 3DRobot
      benchmark.}
    \label{Fig:3DRobotBenchmark}
\end{figure}


\bibliography{citations.bib}{}
\bibliographystyle{unsrt}

\end{document}
