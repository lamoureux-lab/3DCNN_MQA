Proteins are the cogs of the cell machinery. In order to understand responses of a cell to the external stimuli given their internal state, we
have to be able to predict the function of the proteins expressed in a cell. The current paradigm dictates that the function 
of a protein is determined mainly by its structure that dictates its interactions with other molecules present in a cell. 
Therefore the problem of a protein structure prediction (protein folding) is one of the limiting factors in understanding and designing living
organisms.

The progress in the field of protein folding is monitored by Critical Assessment of protein Structure Prediction (CASP) \cite{moult1995large}. This 
is a community-wide experiment to evaluate the protein folding prediction methods. Every method participating in this competition has a pipline
that includes sampling and quality assessment (QA) steps. During the sampling part, the candidate conformations of a protein are generated. The 
QA part of an algorithm tries to select the candidates closest to the unknown native structure.

In this paper we explore the application of deep learning, a new machine learning technique to the protein decoys quality assessment problem. 
Deep learning (DL) is a popular  approach in the field of machine learning, which recently gained considerable interest 
in the research community \cite{lecun2015deep}, particularly in computer vision and image recognition. 
Unlike previous ‘shallow’ approaches, DL tries to learn hierarchical representation of the 
data in hand. It alleviates the need for feature extraction that constituted the bulk of the work done by researchers. 

Recently, DL was applied to biological data and yielded remarkable results in the human splicing code prediction \cite{xiong2015human}, 
identification of DNA- and RNA-binding  motifs \cite{alipanahi2015predicting} and predicting the effects of non-coding 
DNA variants at single nucleotide polymorphism 
precision \cite{zhou2015predicting}. These successes have one thing in common: they do not introduce any features 
between the raw data and the deep learning model. 

\subsection{Related work}
Deep learning methods were also applied in the field of structure quality assessment \cite{}. In particular, 
DeepQA \cite{} used 9 scores from other QA estimators and 
7 physico-chemical feature estracted from a structure as the input features to the deep boltzman machine \cite{}. 
In the DL-PRO algorithm \cite{}, authors first compute contact maps of the decoys and compress them using PCA. 
The vectors from PCA are then fed into an autoencoder to predict the score of the decoy. 
The authors that apply the deep learning methods use them as the ordinary 'shallow' classifiers. 
Therefore they do not get all the advantages these new techniques offer.