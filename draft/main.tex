\documentclass[a4paper,10pt]{article}
\usepackage[utf8]{inputenc}
\usepackage[english]{babel}
\usepackage{float}
\usepackage{graphicx}
\usepackage{caption}
\usepackage{subcaption}
\usepackage{amsmath}
\usepackage{multirow}

%opening
\title{Deep convolutional networks for protein fold quality assessment}
\author{}

\begin{document}

\maketitle

\begin{abstract}

\end{abstract}

\section{Introduction}
% 
The protein folding problem remains one of the outstanding challenges
in structural biology \cite{dill2012folding}.  It is usually defined
as the task of predicting the three-dimensional (3D) structure of a
protein from its amino acid sequence. A crucial step when making such
prediction is the ranking of candidate structures. Given a number of
structural models, can we computationally predict how close each of
them is to the unknown native fold of the protein?

Progress in the field is monitored through the Critical Assessment of
protein Structure Prediction (CASP) competition \cite{moult1995large},
in which protein folding methods are evaluated in terms of their
accuracy at predicting protein structures ahead of their
publication. Most methods in competition include a conformational
sampling step, which generates a number of plausible protein
conformations, and a quality assessment step, which attempts to select
the conformations closest to the unknown native structure.

In this work we explore the application of deep learning to the
problem of ``model quality assessment'' (MQA), also called
``estimation of model accuracy'' (EMA) \cite{kryshtafovych2015}. Deep
learning (DL) has recently garnered considerable interest in the
research community \cite{lecun2015deep}, particularly in computer
vision and natural language processing. Unlike more ``shallow''
machine learning approaches, deep learning improves performance by
learning a hierarchical representation of the raw data at hand. It
alleviates the need for feature engineering, which has traditionally
constituted the bulk of the work done by researchers.

Deep learning has recently been applied to biological data and has
yielded remarkable results for predicting the effects of genetic
variations on human RNA splicing \cite{xiong2015human}, for
identifying DNA- and RNA-binding
motifs \cite{alipanahi2015predicting}, and for predicting the effects
of non-coding DNA variants with single nucleotide
precision \cite{zhou2015predicting}. These successes have one thing in
common: they use raw data directly as input and do not attempt to
engineer features from them.

DL-inspired methods have been used for protein structure quality
assessment \cite{nguyen2014dlpro, cao2016deepqa,
uziela2017proq3d}. For instance, DeepQA \cite{cao2016deepqa} uses 9
scores from other MQA methods and 7 physico-chemical features
extracted from the structure as input features to a deep restricted
Boltzmann machine \cite{hinton2006fast}. The method has been
reported \cite{cao2016deepqa} to outperform ProQ2 \cite{ray2012proq2},
which was the top-performing method in the CASP11
competition \cite{kryshtafovych2015}.  ProQ3D \cite{uziela2017proq3d}
uses the same high-level input features as the earlier ProQ3
method \cite{uziela2016proq3} (including the
Rosetta \cite{leaverfay2011rosetta} score) but achieves better
performance by replacing the support vector machine model by a deep
neural network. Since the original ProQ3 method had one of the top
performances in CASP12 \cite{elofsson2017qacasp12}, it can be expected
that ProQ3D performs equally well. Although both DeepQA and ProQ3D
methods are based on deep neural networks, they use high-level
features as input. In that sense, they use DL models more as
traditional ``shallow'' classifiers than as end-to-end learning
models. It is likely that they do not get all the advantages offered
by the DL approach.
%
By comparison, the DL-Pro algorithm \cite{nguyen2014dlpro} uses a
sligthly more raw input, consisting of the eigenvectors of the
C$\alpha$-to-C$\alpha$ distance matrix. The model itself is an
autoencoder \cite{hinton2006reducing} trained to classify the
structures into either ``near native'' or ``not near native''.

More in line with the ``end-to-end'' spirit of deep learning, methods
using as input a 3D representation of the structure have been
developed to score protein-ligand poses \cite{wallach2015atomnet,
ragoza2017protein}, to predict ligand-binding protein
pockets \cite{jimenez2017deepsite}, and to predict the effect of a
protein mutation \cite{torng2017}. The molecules of interest are
treated as 3D objects represented on a grid and the predictions are
obtained from that information only. While a rigorous comparison of
each of these methods is not always possible, they generally appear to
improve on the state of the art: Both
AtomNet \cite{wallach2015atomnet} and the 3D convolutional neural
network of Ragoza et al.\ \cite{ragoza2017protein} perform
consistently better than either Smina \cite{koes2013smina} or AutoDock
Vina \cite{trott2009vina}.
%
For small molecules, Sch\"{u}tt and coworkers \cite{schutt2017quantum,
schutt2017moleculenet} have developed deep neural networks to predict
the molecular energy of a variety of chemical compounds in various
conformations (or even various isomeric states). These models,
intended to be used as universal force fields, are trained on ab
initio energies and forces, and use only the nuclear charges and the
interatomic distance matrix as input.


\section{Materials and Methods}

\subsection{Datasets}
%
We train and assess our method using the protein decoy datasets from
the CASP competition \cite{moult2014critical}.  We use the CASP7 to
CASP10 data as training set and the CASP11 data as test set, for a
total of 564 target structures in the training set and 83 target
structures in the test set. Each target from the training set has 282
decoys on average.
%
The test dataset is split into two subsets \cite{kryshtafovych2015}:
``stage 1'' with, for each target, 20 decoys selected at random from
all server predictions from the CASP11 competition, and ``stage 2''
with, for each target, the 150 decoys considered best by the
Davis-QAconsensus evaluation method \cite{kryshtafovych2015}.
%
The native structures were not included in the analysis, neither
during the training phase nor during the testing phase. To make the
structural data more consistent, the side chains of all decoy
structures were optimized using the SCWRL4 program
\cite{krivov2009improved}.


The distributions of target sequences lengths are shown on
Fig. \ref{Fig:dataLengthDist}. They confirm that the two datasets
cover the same range of sequence lengths. To ensure that the training
and test sets are significantly different, we have aligned all test
sequences against all training sequences using blastp
\cite{altschul1990basic}. The most significant alignments (E-value${}
< 10^{-4}$) are shown in Table \ref{Tbl:datasetsSimilarity}.  Less
than 11\% of the targets in the test set (9 out of 83) have sequence
similarity with any target in the training set.

To further assess the evolutionary similarity of the two datasets, we
have computed their overlap in terms of Pfam families
\cite{finn2016pfam}. Pfam families were found using HMMER \cite{hmmer}
with an E-value cutoff of 1.0 \cite{finn2016pfam}.  Table
\ref{Tbl:SharedPfam} shows the targets in the test and train sets that
share a family. Accounting for targets for which no Pfam family could
be determined, approximately 25\% of the test set targets share a
family with approximately 10\% of the training set targets.

We have also compared the structures in the training and test sets
using the ECOD database \cite{cheng2014ecod}. This database provides a
five-level classification of all structures of the RCSB PDB
\cite{berman2000protein} according to the following criteria:
architecture (A-group), possible homology (X-group), homology
(H-group), topology (T-group), and family (F-group).  Since the ECOD
classification is domain-based, multi-domain protein chains can belong
to multiple A-, X-, H-, T-, or F-groups.  The higher the level two
protein domains occupy, the more structurally similar they are.
%
Figure \ref{Fig:foldsGraph} shows the classification of the test set
into ECOD groups. Branches drawn in black correspond to groups
containing structures from the training set as well. Branches drawn in
grey correspond to groups unique to the test set. Targets T0773,
T0797, and T0816 are excluded from the analysis because they have no
ECOD classification, and targets T0820, T0823, T0824, T0827, T0835,
and T0836 are excluded because they have no structure in the RCSB PDB.
%
Four architecture groups present overlap at all levels between the
training and test sets: ``a/b barrels'', ``beta duplicates or obligate
multimers'', ``a+b complex topology'', and ``a+b four layers''.
%%% Is this a useful observation? I don't know what to do with that
%%% information. More generally, I really don't know what to make of
%%% Figure 2. It shows a lot of information that we don't need and the
%%% information we need is mainly in Figure 3.

A more detailed view of the overlap between the training and test sets
is presented in Fig. \ref{Fig:summaryTable}. For each target domain in
the test set (T0759 to T0858), a black tile indicates that at least
one structure from the training set belong to the same ECOD group (A,
X, H, T, and F).
%
%A black tile in the ``Family'' row indicates that at least one
%structure from the training set belongs to the same Pfam family as the
%target. (A grey tile indicates that no Pfam family information is
%available for the target.) A black tile in the ``Clan'' row indicates
%similar information for Pfam clans.
%
%Finally, a black tile in the ``Alignment'' row indicates that at least
%one sequence in the training set aligns to the target sequence with an
%E-value smaller than $10^{-4}$.


\begin{figure}[H]
    \centering
    \includegraphics[width=\linewidth]{Fig/datasetLengthDistributions.png}
    \caption{Distributions of sequence lengths for targets in training set (blue) and test set (red).}
    \label{Fig:dataLengthDist}
\end{figure}

\begin{table}[H]
\begin{center}
\begin{tabular}{ c | c | l }
    
    Test set ID & Closest training set ID & E-value \\
    \hline
    T0768 & T0690 & $2.70\times 10^{-13}$ \\
    T0770 & T0645 & $1.79\times 10^{-13}$ \\
    T0772 & T0518 & $1.89\times 10^{-7}$ \\
    T0776 & T0707 & $3.98\times 10^{-5}$ \\
    T0783 & T0699 & $1.19\times 10^{-22}$ \\
    T0798 & T0308 & $9.57\times 10^{-6}$ \\
    T0813 & T0398 & $2.45\times 10^{-5}$ \\
    T0819 & T0636 & $8.66\times 10^{-15}$ \\
    T0854 & T0324 & $2.13\times 10^{-13}$ \\
\end{tabular}
%   
    \caption{Closest homolog sequences from the training set. A
    sequence pair is reported if at least one training sequence aligns
    to a test sequence with a blastp E-value less than $10^{-4}$. Only
    the top alignment is reported for each test sequence.}
%
\label{Tbl:datasetsSimilarity}
\end{center}
\end{table}


\begin{table}[H]
\begin{center}
\begin{tabular}{ l | l | l }

    Common family & Test set target & Train set targets \\
    \hline
    %%% I've removed the version numbers of the Pfam IDs
    %%% I don't think we need them
    PF00795 & T0794 & T0542 \\ \hline
    PF13472 & T0776 & T0448, T0297, T0286, T0750 \\ \hline
    PF03807 & T0813 & T0398, T0393, T0702 \\ \hline
    PF00266 & T0801 & T0339, T0697 \\ \hline
    PF01128 & T0783 & T0699, T0420 \\ \hline
    PF07949 & T0780 & T0572 \\ \hline
    PF13577 & T0815 & T0752, T0736 \\ \hline
    PF12804 & T0783 & T0593, T0699, T0420 \\ \hline
    PF13242 & T0854 & T0371, T0341, T0303, T0324, T0330, T0329, T0418 \\ \hline
    PF13306 & T0768 & T0690, T0671, T0713, T0653 \\ \hline
    PF12741 & T0770 & T0664, T0645, T0532 \\ \hline
    PF00025 & T0798 & T0308 \\ \hline
    PF12872 & T0792 & T0549 \\ \hline
    PF03446 & T0813, T0851 & T0398, T0393, T0702 \\ \hline
    PF00155 & T0801, T0819 & T0591, T0636, T0436, T0697 \\ \hline
    PF13419 & T0854 & T0371, T0341, T0303, T0379, T0324, T0330, T0329, T0418, T0635 \\ \hline
    PF12680 & T0815 & T0451, T0475 \\ \hline
    PF06439 & T0772 & T0518 \\ \hline
    PF12771 & T0770 & T0664, T0645, T0532 \\ \hline
    PF08477 & T0798 & T0308 \\ \hline
    PF00657 & T0776 & T0297, T0286, T0679 \\ \hline
    PF00071 & T0798 & T0308 \\ \hline
    PF00702 & T0854 & T0303, T0324, T0330, T0329, T0418, T0635 \\ \hline
    PF01926 & T0798 & T0308 \\ \hline
    PF12697 & T0764 & T0672 \\ \hline
\end{tabular}
   
\caption{Targets from test and training sets that belong to the same Pfam
family \cite{finn2016pfam}, based on a HMMER search \cite{} with an
E-value cutoff of 1.0. With that cutoff, 403 of the 564 training
targets and 65 of the 83 test targets could be assigned
families. There are 25 families containing at least one test sequence
and one training sequence, involving a total of 16 test targets and 42
training targets. Each of the 25 families belongs to a distinct Pfam
clan.}
%
\label{Tbl:SharedPfam}
\end{center}
\end{table}

\begin{figure}[H]
    \centering
    \includegraphics[width=\linewidth]{Fig/folds_graph.png}
%
    \caption{Classification of the test set structures into the lower
    four ECOD structural levels (from the center out): architecture
    (A), possible homology (X), homology (H), and topology (T). The
    names of the architecture types are shown in the outer circle of
    the diagram.
%%% Why do you have two different architecture names in the same A
%%% group? (brown color: ``alpha complex topology'' and ``alpha
%%% arrays'')
    The grey lines denote test set classes that have no
    respective representative in the training set. The black lines
    show the classes that have representatives in both training and
    test sets. We do not show the F-groups because they have litle
    overlap among the training and test sets.
%%% Why not showing the F-groups?
}
%
    \label{Fig:foldsGraph}
\end{figure}

\begin{figure}[H]
    \makebox[\textwidth]{
    \includegraphics[width=\paperwidth]{Fig/summary_table.png}
    }
%
    \caption{Overlap of the training set on each target domain of the
    test set (from T0759 to T0858). The first 5 rows of tiles
    correspond to the ECOD classification of protein domains (A-, X-,
    H-, T-, and F-groups). A black tile in any of these rows indicates
    that at least one structure from the training set belongs to the
    same ECOD group as the target. Targets for which no ECOD
    classification is available are left empty.
%%% I see that all targets excluded from the analysis have an empty
%%% row of squares. Is T0838 excluded as well? What about the targets
%%% that are not in the list? (775, 778, 779, 791, 793, 795, 799, 802,
%%% 804, 809, 826, 828, 839, 842, 844, 846, 850) Were they all
%%% excluded from the CASP competition? The CASP11 QA paper mentions
%%% that the following targets were cancelled by the organizers: 778,
%%% 779, 791, 809, 842, 844, 846, 850. What about the other ones?
    A black tile in the ``Family'' row indicates that at least one
    structure from the training set belongs to the same Pfam family as
    the target. (A grey tile indicates that no Pfam family information
    is available for the target.) The ``Clan'' row shows similar
    information for Pfam clans. A black tile in the ``Alignment'' row
    indicates that at least one sequence in the training set aligns to
    the target sequence with an E-value smaller than $10^{-4}$.}
%
    \label{Fig:summaryTable}
\end{figure}


\subsection{Input}
%The protein structure is represented as the density maps of eleven atom types. 
We used the types shown in the Table \ref{Tbl:atomTypes}. Initially 21 atom types were proposed by X.Zou et al. 
\cite{huang2006iterative, huang2008iterative} using SYBYL rule set \cite{wang2006automatic}. In this work we clustered similar
types due to the hardware constraints. The density of an atom was modeled using the function: 
$$
\rho(r) =  \begin{cases}
               e^{-\frac{r^2}{2}}&r\leq 2.0\AA~\\
               0                 &r>2.0\AA~\\
            \end{cases}
$$
Each atom density was projected to the grid for the corresponding atom type. The resolution of the grid was set to 1\AA~ and the size of 
each grid to 120x120x120 cells.

\begin{table}[H]
\begin{center}
\begin{tabular}{ c | l | l }
    
    Type & Description & Atoms \\
    \hline
    1 & Sulfur containing atoms & CYSSG, METSD, MSESE \\ \hline
    2 & Amide nitrogens & ASNND2, GLNNE2, backbone N \\ \hline
    3 & Aromatic nitrogens & HISND1, HISNE2, TRPNE1 \\ \hline
    4 & Guanidine nitrogens & ARGNH1, ARGNH2, ARGNE \\ \hline
    5 & Nitrogen with three hydrogens & LYSNZ \\ \hline
    6 & Carboxyl oxygen & ACEO, ASNOD1, GLNOE1, backbone O \\ \hline
    7 & Oxygen in hydroxyl group & SEROG, THROG1, TYROH \\ \hline
    8 & Oxygen in carboxyl group and terminus oxygen & ASPOD1, ASPOD2, GLUOE1, GLUOE2, \\
     & &  O-terminal, OT2-terminal, OXT-terminal \\ \hline
    9 & Sp2 carbon & ARGCZ, ASPCG, GLUCD, ACEC, \\
     & & ASNCG, GLNCD, backbone C \\ \hline
    10 & Aromatic carbon & HISCD2, HISCE1, HISCG, PHECD1 \\
     & & PHECD2, PHECE1, PHECE2, PHECG \\ 
     & & PHECZ, TRPCD1, TRPCD2, TRPCE2 \\
     & & TRPCE3, TRPCG, TRPCH2, TRPCZ2 \\
     & & TRPCZ3, TYRCD1, TYRCD2, TYRCE1 \\
     & & TYRCE2, TYRCG, TYRCZ \\ \hline
    11 & Sp3 carbon & ALACB, ARGCB, ARGCG, ARGCD \\
     & & ASNCB, ASPCB, GLNCB, GLNCG \\
     & & GLUCB, GLUCG, HISCB, HISCB \\
     & & ILECB, ILECD1, ILECG1, ILECG2 \\
     & & LEUCB, LEUCD1, LEUCD2, LEUCG \\
     & & LYSCB, LYSCD, LYSCG, LYSCE \\
     & & METCB, METCE, METCG, MSECB \\
     & & MSECE, MSECG, PHECB, PROCB \\
     & & PROCG, PROCD, SERCB, THRCG2 \\
     & & TRPCB, TYRCB, VALCB, VALCG1 \\
     & & VALCG2, ACECH3, THRCB, CYSCB \\
     & & backbone CA \\ \hline
    
\end{tabular}
    
    \caption {Atom types used in this work. In the atom notation the first three letters are the name of an aminoacid and the rest is 
    the atom name in the PDB format.}
    \label{Tbl:atomTypes}
\end{center}
\end{table}
Figure \ref{Fig:atomic_densities} shows an example of atomic densities of a peptide with the PDB code 5eh6.
Only non-zero densities are shown.

\begin{figure}[H]
    \centering
    \includegraphics[width=\linewidth]{Fig/atomic_densities_V3.png}
    \caption{The example of the representation of a protein using atomic densities. The density map is 
    shown using the volumetric rendering plugin for PyMol. The pdb-code of the protein used for this visualization is 5eh6.
    The isosurface level was set to $0.5$.}
    \label{Fig:atomic_densities}
\end{figure}

\subsection{Model}
%Convolutional neural networks were first proposed for the image recognition by Y.LeCun \cite{lecun1989backpropagation}, 
but gained recognition after the 
ImageNet 2012 competition \cite{krizhevsky2012imagenet}. 
In this work we use 3D convolutional networks to score protein structures. The network consists of 
basic layers that transform the input.
The schematic representation of the model architecture is shown on the Fig \ref{Fig:CNNModel}.
It is comprised of three blocks of alternating convolutional, 
volumetric batch normalization and ReLU layers and three 
fully connected layers with ReLU nonlinearities. The final output of the network is a single number, 
that is interpreted as a score of an input structure. Further we provide a consize description of each layer.

\begin{figure}[H]
    \centering
    \includegraphics[width=\linewidth]{Fig/ConvnetDiagramV1.png}
    \caption{The schematic representation of the convolutional neural network architecture used in this work. 
    The arrows between the boxes denote maximum pooling layers and the connections denote 
    consequent 3d convolutional, batch normalization and ReLU layers. The numbers xM denote the number of filters 
    used in the corresponding 3d convolutional layer. The size of all filters and 
    maximum pooling domains are 3x3x3. The grey stripes denote one-dimensional vectors and crossed lines between them 
    stand for fully-connected layer with ReLU non-linearities. The details of the model parameters can be found in 
    Supplementary Information.}
    \label{Fig:CNNModel}
\end{figure}

The 3D convolutional layer takes $N$ input density maps and transforms them using $M$ filters
according to the following formula:
$$
f^{out}_i (\mathbf{r}) = \sum^{N}_{j=1} \int F_i (\mathbf{r} - \mathbf{\tau}) \cdot f^{in}_j(\mathbf{\tau}) ~d\tau, i \in [1,M]
$$
where filters are denoted using $F_j$. In standard implementations convolutions are approximated by the sum on a grid.
The ReLU nonlinearity is computed in the following way:
$$
f^{out}_i (\mathbf{r}) = \begin{cases}
               f^{in}_i(\mathbf{r})& f^{in}_i(\mathbf{r})\geq 0\\
               0                 &f^{in}_i(\mathbf{r})<0\\
            \end{cases}, i \in [1,M]
$$
The batch normalization layer was introduced by S.Ioffe and C.Szegedy \cite{ioffe2015batch} to address the problem of internal convariate shift:
the change in the distribution of subnetwork outputs due to the change in its parameters during the training. In practice, this layer takes the 
input and normalizes its values according to the mean values and variances of the corresponding values in the subset of examples used to estimate 
the gradient (minibatch):
$$
    \hat{f}^{in}_i(\mathbf{r}) = \frac{f^{in}(\mathbf{r}) - \mu_{B}}{\sqrt{\sigma^{2}_{B} + \epsilon}}, i \in [1,N_B]
$$
where $\mu_B(\mathbf{r}) = \frac{1}{N_B} \sum_{i=1}^{N_B} f^{in}_i(\mathbf{r})$ and 
$\sigma^{2}_{B} = \frac{1}{N_B} \sum_{i=1}^{N_B} \left( f^{in}_i(\mathbf{r}) - \mu_B (\mathbf{r}) \right)$. 
The constant $\epsilon$ is added to avoid division by zero, the number of 
examples in the minibatch is denoted as $N_B$. Afterwards, the output of this layer is computed by scaling the normalized inputs:
$$
f^{out}_i(\mathbf{r}) = \gamma \hat{f}^{in}_i(\mathbf{r}) + \beta, i \in [1,N_B]
$$
The parameters $\gamma$ and $\beta$ are learned along with other parameters of the network during the training.

The maximum pooling layer (MaxPool) is used to build a coarse-grained representation of the input. The output of this layer is 
the maximum over the cubes of the size $d$, that cover the input domain without overlapping. 
The output data domain is then $d$ times smaller than the input data box.

During the coarse-graining procedures the size of the input data box eventually shrinks to a single cell. The flattening layer reshapes the array of 
3d density maps to a single vector.

Afterwards, we compute several transformations using fully-connected layers. These layers transform a vector $\mathbf{x_{in}}$ in the follosing way:
$$
x_{out} = \mathbf{W} \cdot x + \mathbf{b}
$$
where $\mathbf{W}$ is a matrix and $\mathbf{b}$ is a vector, that are learned during the training.


\subsection{Loss functions}

The problem of decoy quality assessment is essentially a ranking problem: we have to arrange decoys according to 
the proximity to the corresponding native structure, which is quantified using GDT score. The ranking problem 
was previously used along with neural networks to 

During the training procedure we load several decoy structures of one target (minibatch) into memory, compute the 
output of the network and the average loss:
$$ L = \frac{1}{N^{2}_B} \sum_{i=1,j=1, i \neq j}^{N_B} L_{ij} $$ 
where $N_B$ is the number of decoys in a minibatch. Afterwards, we compute the gradient of the average loss with respect 
to the network parameters and update them using Adam algorithm.

We used the margin ranking loss for each pair of decoys. Let a decoy representation be denoted as $x_i$ and therefore the output
of the network on this decoy will be $f(x_i)$. Next, let $y_{ij}$ be the ordering coefficient of the two decoys we pick:
$$
y_{ij} = \begin{cases}
               1& gdtts_i \leq gdtts_j \\
               -1& gdtts_i > gdtts_j \\
            \end{cases}
$$
Here $gdtts_i$ is the GDT score of the $i$-th decoy. In principle, any target function can be chosen. 
The pairwise ranking scoring function has the following expression:

$$ L_{ij} = w_{ij} \cdot \max \left[ 0, 1 - y_{ij} \left( f \left( x_i \right) - f \left( x_j \right) \right) \right] $$

the term $w_{ij}$ represents an example weight:

$$
w_{ij} = \begin{cases}
               1& \left| gdtts_i - gdtts_j \right| > threshold \\
               0& otherwise \\ 
            \end{cases}
$$

where the $threshold$ is a constant set to 0.1{\AA}. If the two decoys are too similar, 
we avoid scoring them against each other during the training.



\subsection{Evaluation criteria}
We evaluated our algorithm using the correlation coefficients, Z-score and loss criterions. The correlation coefficents 
were computed between the score of 
our model and GDT-TS metric for all the decoys of each target protein in a test set and then averaged. 
The Z-score is the deviation of the score of 
the best decoy for a certain protein and average decoy score for this protein:
$$ 
Z-score = \frac{f( argmin(gdtts(x_i)) ) - <f(x)>}{std.dev.f(x)}
$$ 
the best decoy is the one with the lowest GDT-TS score. 
The loss criterion is the deviation of the GDT-TS of the best decoys for a protein from the GDT-TS score of the decoy with the lowest score:
$$ 
Loss = | max_i( gdtts_i ) - gdtts( argmin(f(x_i) ) |
$$ 

\subsection{Optimization and dataset sampling}
The optimization procedure of deep convolutional networks usually is stochastic: the function value and gradient 
is estimated on a small subset (batch) of all the training 
examples. We used the batch of size 10 due to the memory limitations. Afterward the parameters of the model are 
changed in the direction of the estimated gradient.
The parameter update step was performed using the Adam algorithm \cite{}. 

The dataset was sampled in the following way: first we chose a random protein from the dataset, then we sample decoys of this protein. 
The procedure is repeated for all the 
proteins in a dataset. One pass through all the proteins in a dataset is called epoch. 
The decoys are sampled in a homogeneous way: we divide all the decoys into $M$ clusters by the value of GDT-TS score. 
Precisely, the decoy $i$ belongs to the cluster  
number $ \left[ \frac{\max(gdtts) - gdtts_i}{\max(gdtts) - \min(gdtts)} \right] + 1$, where $\max(gdtts)$ and $\min(gdtts)$ 
are computed for all the decoys of 
the chosen protein. If there are empty clusters, then we take secon decoys from each non-empty and so on until we filled the batch. 
At the end of each epoch we randomly
shuffle the order of protein and the order of decoys in each cluster. 

Each decoy from the selected batch is randomly rotated and translated. The rotations are sampled uniformly \cite{}. The translation are chosen
in such a way, that the bounding box of the translated protein lies within the box of the size 120x120x120\AA. 

To select the final model we randomly divided the training set into training and validation subsets. The validation subset consists of 
35 targets and their decoys. This subset was not sampled during the training. 
Figure \ref{Fig:TrainingLoss} shows the Kendal tau, Pearsor R coefficients and the loss on the vaidation subset. 
The final model was chosen according to the minimum loss (epoch 40).
\begin{figure}[H]
    \centering
    \includegraphics[width=\linewidth]{Fig/kendall_validation.png}
    \caption{The loss, Kendal tau and Pearson R coefficients evaluated on the validation subset during the training procedure. The epoch 
    denotes that all the targets in the training subset were sampled.}
    \label{Fig:TrainingLoss}
\end{figure}

The table \ref{Tbl:TrainingResults} summarizes the performance metrics on the training and validation sets for the model at epoch 40.

\begin{table}[H]
\begin{center}
\begin{tabular}{ c | c | c | c | c }
    Data subset & Loss & Pearson & Spearmann & Kendall \\
    \hline
    Training set     &0.146 &0.71 &0.61 &0.45 \\
    Validation set   &0.135 &0.71 &0.59 &0.44 \\ \hline

\end{tabular}
  \caption {Results of the model from epoch 40 on the training and validation subsets.}
    \label{Tbl:TrainingResults}
\end{center}
\end{table}

\section{Results}
During the training of the model we randomly sampled rotational and translational degrees of freedom of a decoy structure. Ideally, we 
want the score assigned by the model to a decoy to be invariant under these transformations. However Fig \ref{Fig:ScoreDistribution} 
shows that the distribution
of scores under rotations and translations follows the gaussian distribution. Therefore, to approximate the average score of a decoy we 
sample a score under random translation and rotation 20 times for each decoy in the test set. The final score is then the average of these 
samples.

\begin{figure}[H]
    \centering
    \includegraphics[width=\linewidth]{Fig/sampling_dist.png}
    \caption{The distribution of scores under random translations and rotations of the decoy LOOPP\_TS1 for the target T0342. The distributions 
    os scores under rotations only and translations only are shown with the dashed and solid lines respectively.
    The normal distribution fitted into the sampled scores under both rotations and translations is show with the dashed line. 
    The parameters of the normal distribution are: the average $\mu = 1.38$ and the standard deviation $\sigma = 0.42$.}
    \label{Fig:ScoreDistribution}
\end{figure}


Table \ref{Tbl:TestResults} shows the comparisson of our model with the state of art methods used for the decoy quality assessement. 
To evaluate the performance we used CASP11 stages 1 and 2 datasets. 
These datasets were preprocessed using SCWRL program to optimize side-chains. 

\begin{table}[H]
\begin{center}
\begin{tabular}{ c | c | c | c | c }
    \multicolumn{5}{ c }{Stage 1} \\ \hline

    QA method & Loss & Pearson & Spearmann & Kendall \\
    \hline
    ProQ2   &0.090 &0.643 &0.506 &0.379 \\
    Qprob   &0.097 &0.631 &0.517 &0.389 \\
    \textbf{3DCNN}   &\textbf{0.072} &0.530 &0.415 &0.318 \\
    RWplus  &0.135 &0.536 &0.433 &0.433 \\ \hline
    
    \multicolumn{5}{ c }{Stage 2} \\ \hline
    
    ProQ2   &\textbf{0.058} &0.372 &0.366 &0.256 \\ 
    Qprob   &0.068 &0.381 &0.387 &0.272 \\
    \textbf{3DCNN}     &\textbf{0.064} &\textbf{0.415} &\textbf{0.402} &\textbf{0.283} \\
    RWplus  &0.084 &0.295 &0.314 &0.220 \\ \hline

\end{tabular}
    
    \caption {Results of our method(3DCNN) and the other state-of-art quality assessment programs on the CASP11 dataset Stage 1 and 2.
            Table shows the absolute average values of correlation coefficients. The averaging was performed for each target in the 
            dataset. Afterwards all the values were averaged over all the targets.}
    \label{Tbl:TestResults}
\end{center}
\end{table}

\section{Analysis}


\section{Discussion}

\bibliography{citations.bib}{}
\bibliographystyle{plain}


\end{document}
