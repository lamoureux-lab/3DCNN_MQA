\documentclass[letter,10pt]{article}
\usepackage[utf8]{inputenc}
\usepackage[english]{babel}
\usepackage{float}
\usepackage{graphicx}
\usepackage{caption}
\usepackage{subcaption}
\usepackage{amsmath}
\usepackage{multirow}
\usepackage{color}


\newcommand{\tchanged}[1]{\textcolor{red}{#1}}
\newcommand{\tstrange}[1]{\textcolor{yellow}{#1}}
% \newcommand{\tchanged}[1]{#1}
% \newcommand{\tstrange}[1]{#1}
%opening
\title{Deep convolutional networks for quality assessment of protein folds}
\author{}

\begin{document}

\maketitle

\begin{abstract}
The computational prediction a protein structure from its sequence
generally relies on a method to assess the quality of a protein
model. Most state-of-the-art assessment methods rely on heavily
engineered features to rank candidate models. In this work, we show
that deep convolutional networks can be used to predict the ranking of
decoy structures solely on the basis of their raw three-dimensional
atomic densities, without any feature tuning.  We develop a model that
performs on par with the best algorithms from the literature. This
work also suggests that it is possible to include to the assessment
algorithms other important three-dimensional information such as
electrostatic potential or solute distribution.
\end{abstract}

\section{Introduction}

The protein folding problem remains one of the outstanding challenges
in structural biology \cite{dill2012folding}.  It is usually defined
as the task of predicting the three-dimensional (3D) structure of a
protein from its amino acid sequence. A crucial step when making such
prediction is the ranking of candidate structures. Given a number of
structural models, can we computationally predict how close each of
them is to the unknown native fold of the protein?

Progress in the field is monitored through the Critical Assessment of
protein Structure Prediction (CASP) competition \cite{moult1995large},
in which protein folding methods are evaluated in terms of their
accuracy at predicting protein structures ahead of their
publication. Most methods in competition include a conformational
sampling step, which generates a number of plausible protein
conformations, and a quality assessment step, which attempts to select
the conformations closest to the unknown native structure.

In this work we explore the application of deep learning to the
problem of ``model quality assessment'' (MQA), also called
``estimation of model accuracy'' (EMA) \cite{kryshtafovych2015}. Deep
learning (DL) has recently garnered considerable interest in the
research community \cite{lecun2015deep}, particularly in computer
vision and natural language processing. Unlike more ``shallow''
machine learning approaches, deep learning improves performance by
learning a hierarchical representation of the raw data at hand. It
alleviates the need for feature engineering, which has traditionally
constituted the bulk of the work done by researchers.

Deep learning has recently been applied to biological data and has
yielded remarkable results for predicting the effects of genetic
variations on human RNA splicing \cite{xiong2015human}, for
identifying DNA- and RNA-binding
motifs \cite{alipanahi2015predicting}, and for predicting the effects
of non-coding DNA variants with single nucleotide
precision \cite{zhou2015predicting}. These successes have one thing in
common: they use raw data directly as input and do not attempt to
engineer features from them.

DL-inspired methods have been used for protein structure quality
assessment \cite{nguyen2014dlpro, cao2016deepqa,
uziela2017proq3d}. For instance, DeepQA \cite{cao2016deepqa} uses 9
scores from other MQA methods and 7 physico-chemical features
extracted from the structure as input features to a deep restricted
Boltzmann machine \cite{hinton2006fast}. The method has been
reported \cite{cao2016deepqa} to outperform ProQ2 \cite{ray2012proq2},
which was the top-performing method in the CASP11
competition \cite{kryshtafovych2015}.  ProQ3D \cite{uziela2017proq3d}
uses the same high-level input features as the earlier ProQ3
method \cite{uziela2016proq3} (including the
Rosetta \cite{leaverfay2011rosetta} score) but achieves better
performance by replacing the support vector machine model by a deep
neural network. Since the original ProQ3 method had one of the top
performances in CASP12 \cite{elofsson2017qacasp12}, it can be expected
that ProQ3D performs equally well. Although both DeepQA and ProQ3D
methods are based on deep neural networks, they use high-level
features as input. In that sense, they use DL models more as
traditional ``shallow'' classifiers than as end-to-end learning
models. It is likely that they do not get all the advantages offered
by the DL approach.
%
By comparison, the DL-Pro algorithm \cite{nguyen2014dlpro} uses a
sligthly more raw input, consisting of the eigenvectors of the
C$\alpha$-to-C$\alpha$ distance matrix. The model itself is an
autoencoder \cite{hinton2006reducing} trained to classify the
structures into either ``near native'' or ``not near native''.

More in line with the ``end-to-end'' spirit of deep learning, methods
using as input a 3D representation of the structure have been
developed to score protein-ligand poses \cite{wallach2015atomnet,
ragoza2017protein}, to predict ligand-binding protein
pockets \cite{jimenez2017deepsite}, and to predict the effect of a
protein mutation \cite{torng2017}. The molecules of interest are
treated as 3D objects represented on a grid and the predictions are
obtained from that information only. While a rigorous comparison of
each of these methods is not always possible, they generally appear to
improve on the state of the art: Both
AtomNet \cite{wallach2015atomnet} and the 3D convolutional neural
network of Ragoza et al.\ \cite{ragoza2017protein} perform
consistently better than either Smina \cite{koes2013smina} or AutoDock
Vina \cite{trott2009vina}.
%
For small molecules, Sch\"{u}tt and coworkers \cite{schutt2017quantum,
schutt2017moleculenet} have developed deep neural networks to predict
the molecular energy of a variety of chemical compounds in various
conformations (or even various isomeric states). These models,
intended to be used as universal force fields, are trained on ab
initio energies and forces, and use only the nuclear charges and the
interatomic distance matrix as input.


\section{Materials and Methods}

\subsection{Datasets}

We train and assess our method using the protein decoy datasets from
the CASP competition \cite{moult2014critical}.  We use the CASP7 to
CASP10 data as training set and the CASP11 data as test set, for a
total of 564 target structures in the training set and 83 target
structures in the test set. Each target from the training set has 282
decoys on average.
%
The test dataset is split into two subsets \cite{kryshtafovych2015}:
``stage 1'' with, for each target, 20 decoys selected at random from
all server predictions from the CASP11 competition, and ``stage 2''
with, for each target, the 150 decoys considered best by the
Davis-QAconsensus evaluation method \cite{kryshtafovych2015}.
%
The native structures were not included in the analysis, neither
during the training phase nor during the testing phase. To make the
structural data more consistent, the side chains of all decoy
structures were optimized using the SCWRL4 program
\cite{krivov2009improved}.


The distributions of target sequences lengths are shown on
Fig. \ref{Fig:dataLengthDist}. They confirm that the two datasets
cover the same range of sequence lengths. To ensure that the training
and test sets are significantly different, we have aligned all test
sequences against all training sequences using blastp
\cite{altschul1990basic}. The most significant alignments (E-value${}
< 10^{-4}$) are shown in Table \ref{Tbl:datasetsSimilarity}.  Less
than 11\% of the targets in the test set (9 out of 83) have sequence
similarity with any target in the training set.

To further assess the evolutionary similarity of the two datasets, we
have computed their overlap in terms of Pfam families
\cite{finn2016pfam}. Pfam families were found using HMMER \cite{hmmer}
with an E-value cutoff of 1.0 \cite{finn2016pfam}.  Table
\ref{Tbl:SharedPfam} shows the targets in the test and train sets that
share a family. Accounting for targets for which no Pfam family could
be determined, approximately 25\% of the test set targets share a
family with approximately 10\% of the training set targets.

We have also compared the structures in the training and test sets
using the ECOD database \cite{cheng2014ecod}. This database provides a
five-level classification of all structures of the RCSB PDB
\cite{berman2000protein} according to the following criteria:
architecture (A-group), possible homology (X-group), homology
(H-group), topology (T-group), and family (F-group).  Since the ECOD
classification is domain-based, multi-domain protein chains can belong
to multiple A-, X-, H-, T-, or F-groups.  The higher the level two
protein domains occupy, the more structurally similar they are.
%
Figure \ref{Fig:foldsGraph} shows the classification of the test set
into ECOD groups. Branches drawn in black correspond to groups
containing structures from the training set as well. Branches drawn in
grey correspond to groups unique to the test set. Targets T0773,
T0797, and T0816 are excluded from the analysis because they have no
ECOD classification, and targets T0820, T0823, T0824, T0827, T0835,
and T0836 are excluded because they have no structure in the RCSB PDB.
%
Four architecture groups present overlap at all levels between the
training and test sets: ``a/b barrels'', ``beta duplicates or obligate
multimers'', ``a+b complex topology'', and ``a+b four layers''.
%%% Is this a useful observation? I don't know what to do with that
%%% information. More generally, I really don't know what to make of
%%% Figure 2. It shows a lot of information that we don't need and the
%%% information we need is mainly in Figure 3.

A more detailed view of the overlap between the training and test sets
is presented in Fig. \ref{Fig:summaryTable}. For each target domain in
the test set (T0759 to T0858), a black tile indicates that at least
one structure from the training set belong to the same ECOD group (A,
X, H, T, and F).
%
%A black tile in the ``Family'' row indicates that at least one
%structure from the training set belongs to the same Pfam family as the
%target. (A grey tile indicates that no Pfam family information is
%available for the target.) A black tile in the ``Clan'' row indicates
%similar information for Pfam clans.
%
%Finally, a black tile in the ``Alignment'' row indicates that at least
%one sequence in the training set aligns to the target sequence with an
%E-value smaller than $10^{-4}$.


\begin{figure}[H]
    \centering
    \includegraphics[width=\linewidth]{Fig/datasetLengthDistributions.png}
    \caption{Distributions of sequence lengths for targets in training set (blue) and test set (red).}
    \label{Fig:dataLengthDist}
\end{figure}

\begin{table}[H]
\begin{center}
\begin{tabular}{ c | c | l }
    
    Test set ID & Closest training set ID & E-value \\
    \hline
    T0768 & T0690 & $2.70\times 10^{-13}$ \\
    T0770 & T0645 & $1.79\times 10^{-13}$ \\
    T0772 & T0518 & $1.89\times 10^{-7}$ \\
    T0776 & T0707 & $3.98\times 10^{-5}$ \\
    T0783 & T0699 & $1.19\times 10^{-22}$ \\
    T0798 & T0308 & $9.57\times 10^{-6}$ \\
    T0813 & T0398 & $2.45\times 10^{-5}$ \\
    T0819 & T0636 & $8.66\times 10^{-15}$ \\
    T0854 & T0324 & $2.13\times 10^{-13}$ \\
\end{tabular}
%   
    \caption{Closest homolog sequences from the training set. A
    sequence pair is reported if at least one training sequence aligns
    to a test sequence with a blastp E-value less than $10^{-4}$. Only
    the top alignment is reported for each test sequence.}
%
\label{Tbl:datasetsSimilarity}
\end{center}
\end{table}


\begin{table}[H]
\begin{center}
\begin{tabular}{ l | l | l }

    Common family & Test set target & Train set targets \\
    \hline
    %%% I've removed the version numbers of the Pfam IDs
    %%% I don't think we need them
    PF00795 & T0794 & T0542 \\ \hline
    PF13472 & T0776 & T0448, T0297, T0286, T0750 \\ \hline
    PF03807 & T0813 & T0398, T0393, T0702 \\ \hline
    PF00266 & T0801 & T0339, T0697 \\ \hline
    PF01128 & T0783 & T0699, T0420 \\ \hline
    PF07949 & T0780 & T0572 \\ \hline
    PF13577 & T0815 & T0752, T0736 \\ \hline
    PF12804 & T0783 & T0593, T0699, T0420 \\ \hline
    PF13242 & T0854 & T0371, T0341, T0303, T0324, T0330, T0329, T0418 \\ \hline
    PF13306 & T0768 & T0690, T0671, T0713, T0653 \\ \hline
    PF12741 & T0770 & T0664, T0645, T0532 \\ \hline
    PF00025 & T0798 & T0308 \\ \hline
    PF12872 & T0792 & T0549 \\ \hline
    PF03446 & T0813, T0851 & T0398, T0393, T0702 \\ \hline
    PF00155 & T0801, T0819 & T0591, T0636, T0436, T0697 \\ \hline
    PF13419 & T0854 & T0371, T0341, T0303, T0379, T0324, T0330, T0329, T0418, T0635 \\ \hline
    PF12680 & T0815 & T0451, T0475 \\ \hline
    PF06439 & T0772 & T0518 \\ \hline
    PF12771 & T0770 & T0664, T0645, T0532 \\ \hline
    PF08477 & T0798 & T0308 \\ \hline
    PF00657 & T0776 & T0297, T0286, T0679 \\ \hline
    PF00071 & T0798 & T0308 \\ \hline
    PF00702 & T0854 & T0303, T0324, T0330, T0329, T0418, T0635 \\ \hline
    PF01926 & T0798 & T0308 \\ \hline
    PF12697 & T0764 & T0672 \\ \hline
\end{tabular}
   
\caption{Targets from test and training sets that belong to the same Pfam
family \cite{finn2016pfam}, based on a HMMER search \cite{} with an
E-value cutoff of 1.0. With that cutoff, 403 of the 564 training
targets and 65 of the 83 test targets could be assigned
families. There are 25 families containing at least one test sequence
and one training sequence, involving a total of 16 test targets and 42
training targets. Each of the 25 families belongs to a distinct Pfam
clan.}
%
\label{Tbl:SharedPfam}
\end{center}
\end{table}

\begin{figure}[H]
    \centering
    \includegraphics[width=\linewidth]{Fig/folds_graph.png}
%
    \caption{Classification of the test set structures into the lower
    four ECOD structural levels (from the center out): architecture
    (A), possible homology (X), homology (H), and topology (T). The
    names of the architecture types are shown in the outer circle of
    the diagram.
%%% Why do you have two different architecture names in the same A
%%% group? (brown color: ``alpha complex topology'' and ``alpha
%%% arrays'')
    The grey lines denote test set classes that have no
    respective representative in the training set. The black lines
    show the classes that have representatives in both training and
    test sets. We do not show the F-groups because they have litle
    overlap among the training and test sets.
%%% Why not showing the F-groups?
}
%
    \label{Fig:foldsGraph}
\end{figure}

\begin{figure}[H]
    \makebox[\textwidth]{
    \includegraphics[width=\paperwidth]{Fig/summary_table.png}
    }
%
    \caption{Overlap of the training set on each target domain of the
    test set (from T0759 to T0858). The first 5 rows of tiles
    correspond to the ECOD classification of protein domains (A-, X-,
    H-, T-, and F-groups). A black tile in any of these rows indicates
    that at least one structure from the training set belongs to the
    same ECOD group as the target. Targets for which no ECOD
    classification is available are left empty.
%%% I see that all targets excluded from the analysis have an empty
%%% row of squares. Is T0838 excluded as well? What about the targets
%%% that are not in the list? (775, 778, 779, 791, 793, 795, 799, 802,
%%% 804, 809, 826, 828, 839, 842, 844, 846, 850) Were they all
%%% excluded from the CASP competition? The CASP11 QA paper mentions
%%% that the following targets were cancelled by the organizers: 778,
%%% 779, 791, 809, 842, 844, 846, 850. What about the other ones?
    A black tile in the ``Family'' row indicates that at least one
    structure from the training set belongs to the same Pfam family as
    the target. (A grey tile indicates that no Pfam family information
    is available for the target.) The ``Clan'' row shows similar
    information for Pfam clans. A black tile in the ``Alignment'' row
    indicates that at least one sequence in the training set aligns to
    the target sequence with an E-value smaller than $10^{-4}$.}
%
    \label{Fig:summaryTable}
\end{figure}


\subsection{Input}
The protein structure is represented as the density maps of eleven atom types. 
We used the types shown in the Table \ref{Tbl:atomTypes}. Initially 21 atom types were proposed by X.Zou et al. 
\cite{huang2006iterative, huang2008iterative} using SYBYL rule set \cite{wang2006automatic}. In this work we clustered similar
types due to the hardware constraints. The density of an atom was modeled using the function: 
$$
\rho(r) =  \begin{cases}
               e^{-\frac{r^2}{2}}&r\leq 2.0\AA~\\
               0                 &r>2.0\AA~\\
            \end{cases}
$$
Each atom density was projected to the grid for the corresponding atom type. The resolution of the grid was set to 1\AA~ and the size of 
each grid to 120x120x120 cells.

\begin{table}[H]
\begin{center}
\begin{tabular}{ c | l | l }
    
    Type & Description & Atoms \\
    \hline
    1 & Sulfur containing atoms & CYSSG, METSD, MSESE \\ \hline
    2 & Amide nitrogens & ASNND2, GLNNE2, backbone N \\ \hline
    3 & Aromatic nitrogens & HISND1, HISNE2, TRPNE1 \\ \hline
    4 & Guanidine nitrogens & ARGNH1, ARGNH2, ARGNE \\ \hline
    5 & Nitrogen with three hydrogens & LYSNZ \\ \hline
    6 & Carboxyl oxygen & ACEO, ASNOD1, GLNOE1, backbone O \\ \hline
    7 & Oxygen in hydroxyl group & SEROG, THROG1, TYROH \\ \hline
    8 & Oxygen in carboxyl group and terminus oxygen & ASPOD1, ASPOD2, GLUOE1, GLUOE2, \\
     & &  O-terminal, OT2-terminal, OXT-terminal \\ \hline
    9 & Sp2 carbon & ARGCZ, ASPCG, GLUCD, ACEC, \\
     & & ASNCG, GLNCD, backbone C \\ \hline
    10 & Aromatic carbon & HISCD2, HISCE1, HISCG, PHECD1 \\
     & & PHECD2, PHECE1, PHECE2, PHECG \\ 
     & & PHECZ, TRPCD1, TRPCD2, TRPCE2 \\
     & & TRPCE3, TRPCG, TRPCH2, TRPCZ2 \\
     & & TRPCZ3, TYRCD1, TYRCD2, TYRCE1 \\
     & & TYRCE2, TYRCG, TYRCZ \\ \hline
    11 & Sp3 carbon & ALACB, ARGCB, ARGCG, ARGCD \\
     & & ASNCB, ASPCB, GLNCB, GLNCG \\
     & & GLUCB, GLUCG, HISCB, HISCB \\
     & & ILECB, ILECD1, ILECG1, ILECG2 \\
     & & LEUCB, LEUCD1, LEUCD2, LEUCG \\
     & & LYSCB, LYSCD, LYSCG, LYSCE \\
     & & METCB, METCE, METCG, MSECB \\
     & & MSECE, MSECG, PHECB, PROCB \\
     & & PROCG, PROCD, SERCB, THRCG2 \\
     & & TRPCB, TYRCB, VALCB, VALCG1 \\
     & & VALCG2, ACECH3, THRCB, CYSCB \\
     & & backbone CA \\ \hline
    
\end{tabular}
    
    \caption {Atom types used in this work. In the atom notation the first three letters are the name of an aminoacid and the rest is 
    the atom name in the PDB format.}
    \label{Tbl:atomTypes}
\end{center}
\end{table}
Figure \ref{Fig:atomic_densities} shows an example of atomic densities of a peptide with the PDB code 5eh6.
Only non-zero densities are shown.

\begin{figure}[H]
    \centering
    \includegraphics[width=\linewidth]{Fig/atomic_densities_V3.png}
    \caption{The example of the representation of a protein using atomic densities. The density map is 
    shown using the volumetric rendering plugin for PyMol. The pdb-code of the protein used for this visualization is 5eh6.
    The isosurface level was set to $0.5$.}
    \label{Fig:atomic_densities}
\end{figure}

\subsection{Model}
Convolutional neural networks were first proposed for the image recognition by Y.LeCun \cite{lecun1989backpropagation}, 
but gained recognition after the 
ImageNet 2012 competition \cite{krizhevsky2012imagenet}. 
In this work we use 3D convolutional networks to score protein structures. The network consists of 
basic layers that transform the input.
The schematic representation of the model architecture is shown on the Fig \ref{Fig:CNNModel}.
It is comprised of three blocks of alternating convolutional, 
volumetric batch normalization and ReLU layers and three 
fully connected layers with ReLU nonlinearities. The final output of the network is a single number, 
that is interpreted as a score of an input structure. Further we provide a consize description of each layer.

\begin{figure}[H]
    \centering
    \includegraphics[width=\linewidth]{Fig/ConvnetDiagramV1.png}
    \caption{The schematic representation of the convolutional neural network architecture used in this work. 
    The arrows between the boxes denote maximum pooling layers and the connections denote 
    consequent 3d convolutional, batch normalization and ReLU layers. The numbers xM denote the number of filters 
    used in the corresponding 3d convolutional layer. The size of all filters and 
    maximum pooling domains are 3x3x3. The grey stripes denote one-dimensional vectors and crossed lines between them 
    stand for fully-connected layer with ReLU non-linearities. The details of the model parameters can be found in 
    Supplementary Information.}
    \label{Fig:CNNModel}
\end{figure}

The 3D convolutional layer takes $N$ input density maps and transforms them using $M$ filters
according to the following formula:
$$
f^{out}_i (\mathbf{r}) = \sum^{N}_{j=1} \int F_i (\mathbf{r} - \mathbf{\tau}) \cdot f^{in}_j(\mathbf{\tau}) ~d\tau, i \in [1,M]
$$
where filters are denoted using $F_j$. In standard implementations convolutions are approximated by the sum on a grid.
The ReLU nonlinearity is computed in the following way:
$$
f^{out}_i (\mathbf{r}) = \begin{cases}
               f^{in}_i(\mathbf{r})& f^{in}_i(\mathbf{r})\geq 0\\
               0                 &f^{in}_i(\mathbf{r})<0\\
            \end{cases}, i \in [1,M]
$$
The batch normalization layer was introduced by S.Ioffe and C.Szegedy \cite{ioffe2015batch} to address the problem of internal convariate shift:
the change in the distribution of subnetwork outputs due to the change in its parameters during the training. In practice, this layer takes the 
input and normalizes its values according to the mean values and variances of the corresponding values in the subset of examples used to estimate 
the gradient (minibatch):
$$
    \hat{f}^{in}_i(\mathbf{r}) = \frac{f^{in}(\mathbf{r}) - \mu_{B}}{\sqrt{\sigma^{2}_{B} + \epsilon}}, i \in [1,N_B]
$$
where $\mu_B(\mathbf{r}) = \frac{1}{N_B} \sum_{i=1}^{N_B} f^{in}_i(\mathbf{r})$ and 
$\sigma^{2}_{B} = \frac{1}{N_B} \sum_{i=1}^{N_B} \left( f^{in}_i(\mathbf{r}) - \mu_B (\mathbf{r}) \right)$. 
The constant $\epsilon$ is added to avoid division by zero, the number of 
examples in the minibatch is denoted as $N_B$. Afterwards, the output of this layer is computed by scaling the normalized inputs:
$$
f^{out}_i(\mathbf{r}) = \gamma \hat{f}^{in}_i(\mathbf{r}) + \beta, i \in [1,N_B]
$$
The parameters $\gamma$ and $\beta$ are learned along with other parameters of the network during the training.

The maximum pooling layer (MaxPool) is used to build a coarse-grained representation of the input. The output of this layer is 
the maximum over the cubes of the size $d$, that cover the input domain without overlapping. 
The output data domain is then $d$ times smaller than the input data box.

During the coarse-graining procedures the size of the input data box eventually shrinks to a single cell. The flattening layer reshapes the array of 
3d density maps to a single vector.

Afterwards, we compute several transformations using fully-connected layers. These layers transform a vector $\mathbf{x_{in}}$ in the follosing way:
$$
x_{out} = \mathbf{W} \cdot x + \mathbf{b}
$$
where $\mathbf{W}$ is a matrix and $\mathbf{b}$ is a vector, that are learned during the training.


\subsection{Loss functions}
The problem of decoy quality assessment is essentially a ranking
problem: we have to arrange decoys according to their similarity to
the corresponding native structure, as quantified by the GDT\_TS score
\cite{zemla2001casp4}, for instance. Such a ranking approach to the
problem of model quality assessment has recently been used by the
MQAPRank method \cite{jing2016sorting}, which, however, relies on a
support vector machine model and uses high-level features as
input.

%%% GL: We need a reference for the ``margin ranking loss''
%%% concept. Who used it first?
%G: I cited two papers, one has the exact same expression that we use
% and the other is what they cite as the pioneering work for this loss function
In the present work, we define the loss function in terms of the
margin ranking loss \cite{joachims2002optimizing, gong2013deep} for each pair of decoys.
%%% GL: I'm simplifying the notation a bit, here. We don't really need
%%% the x_i notation, so I've replaced f(x_i) by f_i.
Let $\text{GDT\_TS}_i$ denote the global distance test total score of
decoy $i$ and let $y_{ij}$ be the ordering coefficient of two decoys
$i$ and $j$:
$$
y_{ij} = \begin{cases}
               1& \text{if }\text{GDT\_TS}_i \leq \text{GDT\_TS}_j \\
               -1& \text{if }\text{GDT\_TS}_i > \text{GDT\_TS}_j \\
            \end{cases}
$$
%
In this work we assume that $\text{GDT\_TS}\in [0,1]$.
%%% GL: Why do you write this? What is the actual range of GDT_TS?
%%% Don't leave the reader guessing...
%G: It is [0,1] but sometimes people assume it is [0, 100%]. We just do not use % here.
% There's nothing to guess, in fact. We just precise the notation.
Let $f_i$ denote the output of the network for decoy $i$. We use the
following expression for the pairwise ranking loss function:
%
$$
L_{ij} = w_{ij} \max \left[ 0, 1 - y_{ij} \cdot (f_i - f_j) \right]
$$
%
The term $w_{ij}$ represents the weight of each example and is defined
so that decoys with similar scores are removed from the training:
%
$$
w_{ij} = \begin{cases}
               1& \text{if } \left| \text{GDT\_TS}_i - \text{GDT\_TS}_j \right| > T \\
               0& \text{otherwise} \\ 
            \end{cases}
$$
%
where $T$ is a threshold constant set to 0.1 GDT\_TS unit.
%%% GL: What is a GDT_TS unit??? It's in Angstroms, right?
%G: No it is something between 0 and 1, that's why I precised the interval for this metrics above.

During the training procedure we load $N_\text{B} = 9$ decoy
structures of a given target into memory (a ``batch'') and compute the
output of the network and the average loss:
$$
L = \frac{1}{N_\text{B}^2} \sum_{i=1}^{N_\text{B}}\sum_{\substack{j=1\\j\neq i}}^{N_\text{B}} L_{ij}
$$
Afterwards, we compute the gradient of the average loss with respect
to the network parameters and update them using the Adam algorithm
\cite{kingma2014adam}.


\subsection{Evaluation criteria}
We evaluate the model using various correlation coefficients of the
scores and using the loss criterion. The latter is defined, for any
given protein, as the absolute difference between the GDT\_TS of the
best decoy and the GDT\_TS of the decoy with the lowest predicted
score $f$:
$$ 
\mathrm{Loss} = \left| \mathrm{max}_i(\text{GDT\_TS}_i) - \text{GDT\_TS}_{\mathrm{argmin}_i(f_i)} \right|
$$
%
The correlation coefficients between the $f$ score produced by the
model and the GDT\_TS score are computed for all decoys of a given
target in the test set, and are then averaged over all targets
(per-target average). Since the value of GDT\_TS increases with the
quality of a model but the value of $f$ decreases, an ideal QA
algorithm would show a correlation coefficient of $-1$ and a loss of
$0$.


\subsection{Optimization and dataset sampling}
The parameter optimization of the model was performed using the Adam
algorithm \cite{kingma2014adam}. The gradient of the loss function
with respect to the model parameters was computed on the pairs of
models in the batch. The batch size was set to $N_\text{B} = 9$
models.
%%% GL: We've introduced the notation N_B for the batch size earlier
%%% on. Moreover, variable M was used earlier for the number of
%%% filters.

%%% GL: Avoid ``we'' and ``our'' as much as possible. It sometimes
%%% sounds more lively but it usually give the results an air of
%%% subjectivity, as if somebody else would find different
%%% results. See what I did to the following paragraph:
The training dataset is sampled by first choosing a random target from
the dataset, then sampling decoys of this target. One epoch
corresponds to one pass through all targets in the dataset. The decoys
are sampled in a homogeneous way, by dividing all decoys of a given
target into $N_\text{B} = 9$ bins by the value of their GDT\_TS score
and picking one decoy from each bin at random.
%
Precisely, decoy $i$ belongs to the bin number 
$$
1 + \left[ N_B \times \frac{\text{GDT\_TS}_i - \min(\text{GDT\_TS}) }{\max(\text{GDT\_TS}) - \min(\text{GDT\_TS})} \right]
$$
where $\max(\text{GDT\_TS})$ and $\min(\text{GDT\_TS})$ are computed on all
decoys of the chosen target.
%%% GL: I repeat my comment: This isn't the formula for the bin
%%% number. Don't you need an integer from 1 to 9?
%G: yep, corrected
During the training we pick $N_B$ decoys from the bins ${1 \dots N_B}$.
If a bin $k$ is empty, the decoy is picked from a non-empty bin $j \neq k$ chosen randomly.
The order of targets and the order of decoys in the bins are randomly shuffled at the end of each
epoch.


Decoy structures are randomly rotated and translated each time they
are used as input. The rotations are sampled uniformly
\cite{shoemake1992uniform} and the translation are chosen in such a
way that the translated protein fits inside the $120$~\AA${}\times
120$~\AA${}\times 120$~\AA\ input grid (see Supporting Information for
details).

We select the final model based on its performance on a
\emph{validation subset} consisting of 35 targets (and their decoys)
picked at random from the training set and excluded from the training
procedure. The remaining 529 targets are called the \emph{training
subset}.  Figure~\ref{Fig:TrainingLoss} shows the Kendall $\tau$ and
Pearson $R$ coefficients and the loss on the validation subset over 52
epochs of training.  Models are saved every 10 epochs and we pick the
one that has the smallest loss (at epoch 40).
%
Table \ref{Tbl:TrainingResults} summarizes the performance metrics on
the training and validation sets for the model at epoch 40.

\begin{figure}[H]
    \centering
    \includegraphics[width=\linewidth]{Fig/kendall_validation.eps}
%
    \caption{Loss, Kendall $\tau$, and Pearson $R$ coefficients
      evaluated on the validation subset during the training
      procedure.  One epoch corresponds to a cycle over all targets in
      the training subset. Models are saved every 10 epochs and the
      arrow shows the minimum validation loss for which a model was
      saved (at epoch 40).}
%
    \label{Fig:TrainingLoss}
\end{figure}

\begin{table}[H]
\begin{center}
\begin{tabular}{ c | c | c | c | c }
    Data & Loss & Pearson $R$ & Spearman $\rho$ & Kendall $\tau$ \\
    \hline
    Training subset     &0.146 &0.71 &0.61 &0.45 \\
    Validation subset   &0.135 &0.71 &0.59 &0.44 \\ \hline
\end{tabular}
    \caption {Performance of the model from epoch 40 on the training
      and validation subsets.}
    \label{Tbl:TrainingResults}
\end{center}
\end{table}


\section{Results}
Ideally, the score assigned by the model to a decoy should not depend
on its position and orientation.  To allow the model to learn this
invariance, we have randomly sampled the rotational and translational
degrees of freedom of all decoy structures during the training.
%
Figure~\ref{Fig:DecoysScoreDistribution} shows the distributions of
scores for several decoy structures of the same target (T0832),
calculated for 900 rotations and translations sampled uniformly.
While the score of a given structure is not strictly invariant under
rotation and translation, it has a relatively narrow, unimodal
distribution.
More importantly, the difference between the average scores of two
decoys is usually larger than their variances. To reduce the influence
of the choice of rotation and translation on the final ranking, we
estimate the score of each decoy from the average of 90 scores
calculated for random rotations and translations.

\begin{figure}[H]
    \centering
    \includegraphics[width=\linewidth]{Fig/decoys_sampling_dist.eps}
%
    \caption{Distributions of the scores of five decoys for target
    T0832 under random translations and rotations. A higher score
    represents a lower quality.}
%
    \label{Fig:DecoysScoreDistribution}
\end{figure}

Table~\ref{Tbl:TestResults} shows a comparison of our model (3DCNN)
with the state-of-the-art model quality assessement methods: ProQ2D,
ProQ3D~\cite{uziela2017proq3d}, VoroMQA~\cite{olechnovivc2017voromqa}
and RWPlus~\cite{zhang2010novel}.
ProQ2D uses carefully tuned features based on surface area accessibilities, residue-residue contacts, 
predicted and observed seconday structure, residue conservation and atomic contacts. These features are 
then used as the inputs of deep neural network to predict local and global structure quality. 
ProQ3D employs the same features as ProQ2D as well as Rosetta energy terms based on full-atom and centroid models of the protein. 
This algorithm also uses the features as an input to the deep neural network to predict the 
local and global quality of the candidate structure.
VoroMQA uses knowledge-based potential that depends on the contact surface between two atoms or the solvent. The contact surfaces are calculated 
by representing atoms as spheres with the radii equal to the Van-der-Waals radii of the corresponding atoms and using Voronoi tesselation to obtain 
the contact surface. 
RWPlus is a pairwise knowledge-based potential that uses freely-joined chain model to calculate distance distributions in the reference state.
We chose ProQ2D and ProQ3D as the methods that are derived from the best performing single-model algorithm in CASP11 (ProQ2). The RWPlus 
was selected because it represents previously widespread knowledge-based approach to separate decoys from the corresponding native structures.
VoroMQA was selected, because this approach is rather dissimilar to the common machine-learning techniques and pairwise distance-based scoring potentials.
The other important criterion for selecting these methods is the availability of their source-code or executable. This allowed us to re-evaluate them on 
our CASP11 benchmark, where the decoys side chains were optimized using SCWRL4 program.

We removed targets T0797, T0798, T0825 from this benchmark, because they were released for multimeric prediction.

\begin{table}[H]
\begin{center}
\begin{tabular}{ c | c | c | c | c }
    MQA method & Loss & Pearson $R$ & Spearmann $\rho$ & Kendall $\tau$ \\ \hline
    \multicolumn{5}{ c }{Stage 1} \\ \hline
    ProQ3D   &0.046 &0.755 &0.673 &0.529 \\
    ProQ2D   &0.064 &0.729 &0.604 &0.468 \\
    \textbf{3DCNN} &0.064 &0.535 &0.425 &0.325 \\    
    VoroMQA  &0.087 &0.637 &0.521 &0.394 \\
    RWplus   &0.122 &0.512 &0.402 &0.303 \\ \hline
    
    \multicolumn{5}{ c }{Stage 2} \\ \hline
    VoroMQA  &0.063 &0.457 &0.449 &0.321 \\ 
    \textbf{3DCNN} &0.064 &0.421 &0.409 &0.288 \\
    ProQ3D   &0.066 &0.452 &0.433 &0.307 \\
    ProQ2D   &0.072 &0.437 &0.422 &0.299 \\
    RWplus   &0.089 &0.206 &0.248 &0.176 \\ \hline

\end{tabular}
%
    \caption{Performance comparison of our method (3DCNN) with other
    state-of-the-art model quality assessment methods on the CASP11
    dataset stages~1 and 2 (see text). The table reports the absolute,
    per-target average values of the correlation coefficients.}
    \label{Tbl:TestResults}
\end{center}
\end{table}



\section{Analysis}
Deep neural networks are often thought of as ``black boxes'' that are
easy to train but difficult to intepret. Interpreting the results
obtained using these techniques may indeed be more challenging than
getting the results themselves. This difficulty of interpretation
sometimes leads to the suspiction that the neural network learns
artifacts in the data that correlate with the target results but are
not meaningful on any datasets other that those, chosen for training and evaluation.
%
In this section we attempt to show that our network learns relevant
description of a protein structure and can successfully rank decoys in the 3DRobot dataset.

First, we identify the regions of a structure that are responsible for
an increase of its score (a \emph{decrease} in its quality). If the
network has learned intepretable features of the data, we expect this
analysis to show that the regions responsible for the poor quality of
a decoy structure are those in which it deviates from the native
structure.
%
We use the Grad-CAM analysis technique proposed by Selvaraju et
al.~\cite{selvaraju2016grad}. The key idea of this technique is to
compute the gradient of the model output with respect to the output of a certain layer (``activations'' of the layer) 
and take the sum of these activations weighted by the gradient.
This highlights output regions of that layer that are both strongly
activated and highly influential on the output.
The weighted layer output is then upscaled to the same size as the input of the network. 
It indicate which parts of the input contribute the most to the gradient of the network output.
In our case we choose the ReLU layer 10, for which
the output size is $25\times 25\times 25$.
We tested the method on the outputs of the neighboring layers and
layer 10 represents the best tradeoff between interpretability and
coarseness.

An example of the Grad-CAM output is given in Fig.~\ref{Fig:GradCAMT0776}.
In line with our scoring procedure, we perform the Grad-CAM analysis
by uniformly sampling 90 rotations and translations of the decoy. We
obtain the Grad-CAM output for each transformation and
project it onto the atoms of the decoy. 
Figure~\ref{Fig:GradCAMT0776} shows two representations of Grad-CAM result 
(for decoy Distill\_TS3 of target T0776): the
projection of the map onto the atoms of the decoy, represented as a
color-coded value on the cartoon rendering of the structure, and the
projection onto a 3D grid, represented as an isosurface. The
isosurface of Fig.~\ref{Fig:GradCAMT0776} is a hollow shell containing
most of the solvent exposed region of the decoy, which indicates that
the network enforces packing: any increase in the atomic density
around the well-packed core decreases the quality of the decoy.
%
We also found that the weighted layer outputs are mostly zero for
structures close to the native ones (results not shown), despite the
fact that no gradient information was included in the training
procedure.

Figure~\ref{Fig:GradCAMT0776_more} shows the color-coded
values of the Grad-CAM result for the four decoys of target T0776.
We see, that the position of the N-terminal alpha-helix (which was not present in the target native structure) 
strongly influences the predictions. In case of the decoys Distill\_TS3 and 3D-Jigsaw-V5\_1\_TS2 our algorithm 
highlights this helix as the one contributing to the quality. On the other hand FALCON\_TOPO\_TS3 places this helix 
differently. This placement also better corresponds to placement of the preceeding loop in the native structure.

\begin{figure}[H]
    \centering
    \includegraphics[width=0.5\linewidth]{Fig/FigT0776.eps}
%
    \caption{Left: The output of the Grad-CAM algorithm for the layer 10 on
    candidate structure Distill\_TS3 of target T0776. The isosurface
    shows the scaled outputs at the two-sigma level. The intensities
    of the outputs at the positions of the protein atoms are
    color-coded on the cartoon rendering of the structure, from blue
    (low intensity) to red (high intensity). Right: Cartoon
    representation of the Distill\_TS3 decoy structure (in blue)
    aligned to the native structure (in red).}
    \label{Fig:GradCAMT0776}
\end{figure}

\begin{figure}[H]
    \centering
    \includegraphics[width=\linewidth]{Fig/T0776.eps}
%
    \caption{The output of the Grad-CAM algorithm for the layer 10 of the network
    projected onto the atoms of the decoys. Each decoy is aligned to
    the native structure, shown as a transparent, blue cartoon.}
    \label{Fig:GradCAMT0776_more}
\end{figure}

To verify that the network we trained does not rely on artifacts in
the data to rank decoys, we have assessed its performance on a second,
independent dataset generated by the 3DRobot
algorithm \cite{deng20163drobot}. The decoys generated by this
algorithm are uniformly distributed within RMSD range of $[0; 12\AA]$
of the native structure and are optimized for the number of hydrogen
bonds and compactness.

The dataset consists of 300 decoys for each of the 200 non-redundant proteins. The proteins were 
selected from the PDB database with less than 20\% pair sequence identity, containing 48 $\alpha$, 
40 $\beta$ and 112 $\alpha/\beta$ signle-domain proteins with length from 80 to 250 residues. 
The absolute Pearson $R$ coefficient averaged per-target over all the
targets in this benchmark was $0.85$. Spearman $\rho$ coefficient
and Kendall $\tau$ coefficient were $0.83$ and $0.64$, respectively.
Representative examples of score versus GDT\_TS plots are shown in
Fig.~\ref{Fig:3DRobotBenchmark}.  We see, that the MQA method devised
in this work successfully ranks unrelated datasets.
\begin{figure}[H]
    \centering
    \includegraphics[width=\linewidth]{Fig/3DRobot_set_sFinal_funnels.eps}
%
    \caption{The plots of score versus GDT\_TS for four best correlated targets and four
    least correlated targets in the 3DRobot benchmark.}
    \label{Fig:3DRobotBenchmark}
\end{figure}


\section{Discussion}

In this work we showed that it is possible to construct an algorithm that learns to assess the quality of protein models from 
raw representation of the model. Our work used the atom types densities as such representation, however it is clear that 
any other physical quantity defined on the grid can be employed. The examples of such quantites are electrodymanic potential and 
water oxygen density distribution. So far, no other quality assessment method was able to generalize to include these 
crucial properties of the solute.

This work also identified some important issues: only approximate score invariance under transformations and the difficulty of 
interpretation of the results. The invariance problem can be solved using the recently published approach by Worral et al. \cite{worrall2016harmonic}.
They restricted the space of coefficient of the convolutional filters to the circular harmonics to attain eqivariance of transformations at each layer 
of the network under rotations. The layerwise equivariance then leads to the invariance of the final output.

The interpretation difficulty of deep neural networks is still an important research problem in machine learning. However, the recently 
published approach to quantify interpretability \cite{bau2017network} of these methods somewhat aleviates this issue. The authors of this 
approach to interpret a network representation used the exhaustively labeled image dataset, that contains the bounding boxes and labels for 
fine-grained features, such as body parts or car parts. In the case of protein models such labels are readily available: 
the secondary structure elements, amino-acids, hydrogen bonding network and disulfide bonds, as well as others do not need to be labels by hand.

Alltogether, we believe that this work opens the way to directly use the information about protein environment, aleviates the need to 
tediously engineer features and shows that the drawbacks of this approach can be solved with the further research. To accelerate the 
mentioned process we publish the code as well as the preprocessed datasets along with this paper \cite{}.


\bibliography{citations.bib}
\bibliographystyle{unsrt}


\end{document}
